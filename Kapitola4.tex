\chapter{Konvoluce klasick\'ych funkc\'i}




\subsection{Pozn\'amka}
Operace konvoluce na prostoru klasick\'ych funkc\'i -- definice na $\LOC(\Er)$
\subsection{V\v eta}
o existenci konvoluce v $\LLL_1(\Er)$
\subsection{Pozn\'amka}
Bilinearita konvoluce v $\LLL_1(\Er)$ (ve cvi\v cen\'i)
\subsection{Pozn\'amka}
Komutativita konvoluce v $\LLL_1(\Er)$
\subsection{Pozn\'amka}
o konvoluci funkc\'i tvaru $\Theta(x)F(x),$ kde $F(x)\in\LOC(\R)$
\subsection{Pozn\'amka}
definice pojm\r u hustota a hustota pravd\v epodobnosti







\section{Konvoluce funkc\'i}
Prezentovanou teorii pravd\v epodobnosti nyn\'i zu\v zitkujeme p\v ri specifick\'em zaveden\'i pojmu konvoluce funkc\'i. Nejprve p\v redstav\'ime tuto operaci pro hustoty pravd\v epodobnosti, a pot\'e tuto definici roz\v s\'i\v r\'ime na co nej\v sir\v s\'i t\v r\'idu funkc\'i.

\subsection{V\v eta}\label{odvozeni_konvoluce}

Nech\v t jsou d\'any nez\'avisl\'e jednorozm\v ern\'e n\'ahodn\'e veli\v ciny $\X,\Y$ s absolutn\v e spojit\'ym rozd\v elen\'im. Nech\v t $f_\X(x)$ a $f_\Y(y)$ jsou p\v r\'islu\v sn\'e hustoty pravd\v epodobnosti. Pak hustotou pravd\v epodobnosti $f_\ZZZ(z)$ n\'ahodn\'e veli\v ciny $\ZZZ=\X+\Y$ je funkce \BE f_\ZZZ(r)= \int_{-\infty}^\infty f_\X(x)f_\Y(r-x)~\dx. \label{konvicka2} \EE

\Proof

\begin{itemize}
\item ozna\v cme $F_\ZZZ(z)$ distribu\v cn\'i funkci n\'ahodn\'e veli\v ciny $\ZZZ=\X+\Y$
\item pro ni plat\'i
%
$$F_\ZZZ(z)=\prob\bigl[\ZZZ \leq z\bigr]=\prob\bigl[\X+\Y \leq z\bigr]=\iint_{x+y \leq z} f_{\X,\Y}(x,y)~\dx\dy$$

\item ozna\v cme $M_z=\bigl\{(x,y)\in\E^2:~x+y\leq z\bigr\}$
\item mno\v zina $M_z$ p\v redstavuje polorovinu v $\E^2$
\item u\v zijeme-li d\'ale p\v redpokladu, \v ze $\X,\Y$ jsou nez\'avisl\'e, tak platí rovnost
%
$$F_\ZZZ(z)=\iint_{M_z} f_\X(x)f_\Y(y)~\dx~\dy= \int_{-\infty}^\infty \int_{-\infty}^{z-x} f_\X(x)f_\Y(y)~\dy~\dx= \left|\begin{array}{c} r=x+y\\ \dr=\dy \end{array}\right|=$$
%
$$= \int_{-\infty}^\infty \int_{-\infty}^{z} f_\X(x)f_\Y(r-x)~\dr~\dx= \int_{-\infty}^{z} \left(\int_{-\infty}^\infty f_\X(x)f_\Y(r-x)~\dx \right)\dr= \int_{-\infty}^{z} f_\ZZZ(r)~\dr$$
\item proto je hledanou hustotou pravd\v epodobnosti funkce $f_\ZZZ(r)= \int_{-\infty}^\infty f_\X(x)f_\Y(r-x)~\dx$

\end{itemize}

\subsection{Pozn\'amka}

Analogicky lze uk\'azat, \v ze pro nez\'avisl\'e v\'icerozm\v ern\'e n\'ahodn\'e veli\v ciny $\XX,\YY$ a $\ZZZZ=\XX+\YY$ plat\'i vztah
%
\BE f_{\ZZZZ}(\rrr)= \int_{\E^r} f_{\XX}(\x)f_{\YY}(\rrr-\x)~\dxx. \label{konvicka} \EE

\subsection{Pozn\'amka}

Vztah \eqref{konvicka} je jedn\'im ze z\'akladn\'ich vztah\r u cel\'e teorie o \v re\v sen\'i parci\'aln\'ich diferenci\'aln\'ich rovnic. Jeho platnost nebudeme zu\v zovat pouze na p\v r\'ipad hustot pravd\v epodobnosti, ale zobecn\'ime ho pro obecn\'e v\'icerozm\v ern\'e funkce.

\subsection{Definice}\label{klasa-konvo}
Nech?jsou dány funkce $f(\x),g(\x)\in \LOC{(\Er)}$. Zobrazení
$(f\star g)(\x):\LOC{(\Er)} \times \LOC{(\Er)}\mapsto \LOC{(\Er)}$
definované předpisem
%
$$(f\star g)(\x):=\int_{\Er}f(\sss)g(\x-\sss)~\dss$$
%
nazveme \emph{konvoluc\'i}\index{konvoluce klasick\'ych funkcí} funkcí, pokud pravá strana existuje a
patří do t\v r\'idy $\LOC{(\Er)}.$

\subsection{Věta}\label{existuje_konvo}

Nech?jsou dány funkce $f(\x),g(\x)\in \LLL_1{(\Er)}$. Pak jejich konvoluce $(f\star g)(\x)$ existuje pro skoro v\v sechna $\x\in\Er$ a nav\' ic patří do t\v r\'idy $\LLL_1{(\Er)}.$ D\'ale 
%
\BE \int_{\Er} (f\star g)(\x)~\dxx \leq \|f\|_{\LLL_1} \cdot  \|g\|_{\LLL_1}. \label{zacapa} \EE

\Proof

\begin{itemize}
\item vyjdeme z p\v redpoklad\r u, \v ze $\int_{\Er} |f(\x)|~\dxx \in\R$ a $\int_{\Er} |g(\x)|~\dxx \in\R$

\item chceme uk\'azat, \v ze tak\'e $\int_{\Er} \bigl|(f\star g)(\x)\bigr|~\dxx \in\R$

\item zkoumejme proto integr\'al $$\int_{\Er} \bigl|(f\star g)(\x)\bigr|~\dxx=$$ $$=\int_{\Er} \left|\int_{\Er}f(\sss)g(\x-\sss)~\dss ~\right|~\dxx \leq \int_{\Er} \int_{\Er} \bigl|f(\sss)g(\x-\sss)\bigr|~\dss~\dxx =\int_{\Er} \left(\int_{\Er} \bigl|g(\x-\sss)\bigr|~\dxx\right) |f(\sss)|~\dss =$$
%
$$=  \left|\begin{array}{c} \y = \x -\sss\\ \dyy=\dss \end{array}\right|= \int_{\Er} \left(\int_{\Er} \bigl|g(\y)\bigr|~\dyy\right) |f(\sss)|~\dss= \int_{\Er} \bigl|g(\y)\bigr|~\dyy \cdot  \int_{\Er} \bigl|f(\x)\bigr|~\dxx \in \R$$

\item podle tvrzen\' i Fubiniovy v\v ety plat\'i, \v ze $f(\x-\y)g(\y)\in\LLL_1(\Er)$ pro skoro v\v sechna $\x\in\Er,$ a tedy konvoluce $(f\star g)(\x)$ je definov\'ana pro skoro v\v sechna $\x\in\Er$
    
\item a proto\v ze $\|h\|_{\LLL_1}:= \int_{\Er} |h(\x)|~\dxx,$ vych\'az\' i z p\v redchoz\' ich \' uvah tak\'e platnost vztahu (\ref{zacapa})

\item vyu\v z\'iv\'ame p\v ritom v\v ety 4.2.37, 4.3.5 a \ 4.3.7 ze skript \cite{Krbalek_MAB4}

\item pro  funkce nez\'aporn\'e s.v. nav\' ic plat\' i ve vztahu  (\ref{zacapa}) rovnost, tj. $\|f\star g\|_{\LLL_1} = \|f\|_{\LLL_1} \cdot  \|g\|_{\LLL_1}$

\end{itemize}


\subsection{Věta}

Nech?jsou dány hustoty pravděpodobnosti $f(\x),g(\x):\Er\mapsto\R$. Pak jejich konvoluce $(f\star g)(\x)$ existuje a nav\' ic je také hustotou pravděpodobnosti. 

\Proof

\begin{itemize}


\item o hustotách $f(\x), g(\x)$ víme, že patří do $\LLL_1(\Er)$ a chceme ukázat, že do $\LLL_1(\Er)$ paří také jejich konvoluce, a navíc, že tato konvoluce je také hustotou pravděpodobnosti

\item Díky nerovnosti $$(f\star g)(\x)=\int_{\Er}f(\sss)g(\x-\sss)~\dss\geq0$$
víme, že je splněna nezápornost hustoty

\item ověřme, že $\int_{\Er}(f\star g)(\x)=1$

$$\int_{\Er}\int_{\Er}f(\sss)g(\x-\sss)\dxx\dss = \int_{\Er}f(\sss)\int_{\Er}g(\x-\sss)\dxx\dss = \int_{\Er}f(\sss)\dss\int_{\Er}g(\y)\dyy=1$$

přičemž jsme v první rovnosti použili Fubiniovu větu a ve druhé rovnosti substituci $\y=\x-\sss$

\item integrály $\int_{\Er}f(\sss)\dss$ a $\int_{\Er}g(\y)\dyy$ jsou z definice hustoty rovny jedné a tedy $(f\star g)(\x)$ je opravdu také hustotou pravděpodobnosti 
\end{itemize}

\subsection{P\v r\'iklad}\label{Meklenburg}

Nech\v t $$f(x) =
\frac{1}{\sqrt{2\pi}\sigma_1}\e^{-\frac{(x-\mu_1)^2}{2\sigma_1^2}},\quad g(x) =
\frac{1}{\sqrt{2\pi}\sigma_2}\e^{-\frac{(x-\mu_2)^2}{2\sigma_2^2}}.$$ Vypo\v ct\v eme konvoluci $f(x)~\star~g(x).$ Z definice konvoluce a ze vztahu $\int_{\R}e^{ax^2}\dx=\sqrt{\frac{\pi}{a}}$ plynou následující rovnosti.
%
\begin{multline*}
f(x)~\star~g(x)=\frac{1}{2\pi\sigma_1\sigma_2}\int_\R  \e^{-\frac{(s-\mu_1)^2}{2\sigma_1^2}}\e^{-\frac{(x-s-\mu_2)^2}{2\sigma_2^2}}~\ds=\left|\begin{array}{c} y=s-\mu_1\\ \dy=\ds \end{array}\right|=\frac{1}{2\pi\sigma_1\sigma_2}\int_\R  \e^{-\frac{y^2}{2\sigma_1^2}}\e^{-\frac{(x-y-\mu_1-\mu_2)^2}{2\sigma_2^2}}~\dy{}=\\
=\left|\begin{array}{c} \lam :=x-\mu_1-\mu_2 \end{array}\right|= \frac{1}{2\pi\sigma_1\sigma_2}\int_\R  \exp\left[-\frac{\sigma_2y^2+\sigma_1(y-\lam)^2}{2\sigma_1^2\sigma_2^2}\right]\dy{}=\\
=\frac{1}{2\pi\sigma_1\sigma_2}\int_\R  \exp\left[-\frac{\sigma_1^2+\sigma_2^2}{2\sigma_1^2\sigma_2^2}\left(\bigl(y-\frac{\sigma_1^2}{\sigma_1^2+\sigma_2^2}\lam\bigr)^2+\lam^2\frac{\sigma_1^2\sigma_2^2}{(\sigma_1^2+\sigma_2^2)^2}\right)\right]\dy {}=\\
=\frac{1}{2\pi\sigma_1\sigma_2}~\e^{-\frac{\lam^2}{2(\sigma_1^2+\sigma_2^2)}}\int_\R  \exp\left[-\frac{\sigma_1^2+\sigma_2^2}{2\sigma_1^2\sigma_2^2}\left(y-\frac{\sigma_1^2}{\sigma_1^2+\sigma_2^2}\lam\right)^2\right]\dy = \frac{1}{2\pi\sigma_1\sigma_2}~\e^{-\frac{\lam^2}{2(\sigma_1^2+\sigma_2^2)}}\int_\R  \e^{-\frac{\sigma_1^2+\sigma_2^2}{2\sigma_1^2\sigma_2^2}z^2}\dz{}=\\
= \frac{1}{2\pi\sigma_1\sigma_2}~\e^{-\frac{\lam^2}{2(\sigma_1^2+\sigma_2^2)}} \sqrt{\frac{\pi}{\frac{\sigma_1^2+\sigma_2^2}{2\sigma_1^2\sigma_2^2}}}= \sqrt{\frac{\pi}{2(\sigma_1^2+\sigma_2^2)}}~ \e^{-\frac{(x-\mu_1-\mu_2)^2}{2(\sigma_1^2+\sigma_2^2)}}.
\end{multline*}
%
Konvoluc\'i dvou hustot pravd\v epodobnosti Gaussova rozd\v elen\'i je tedy podle dosa\v zen\'eho v\'ysledku op\v et hustota pravd\v epodobnosti Gaussova rozd\v elen\'i. Maj\'i-li vstupuj\'ic\'i hustoty st\v redn\'i hodnoty po \v rad\v e $\mu_1,$ $\mu_2$ a rozptyly $\sigma_1^2,$ $\sigma_2^2,$ m\'a v\'ysledn\'a konvoluce st\v redn\'i hodnotu $\mu_1+\mu_2$ a rozptyl $\sigma_1^2+\sigma_2^2.$ Univerzalitu tohoto tvrzen\'i prok\'a\v zeme v n\'asleduj\'ic\'ich v\v et\'ach.

\subsection{Věta}
Nech?$\X,$ resp. $\Y$ jsou nezávislé náhodné veličiny maj\'ic\'i absolutn\v e spojit\'e rozd\v elen\'i. Nech\v t jejich hustoty pravděpodobnosti jsou $f_\X(x)\in\LLL_1(\R),$ resp. $g_\Y(y)\in\LLL_1(\R)$ a nav\'ic $\la x \ra=\mu_1$ a
$\la y \ra=\mu_2.$ Potom hustotou pravděpodobnosti náhodné veličiny
$\ZZZ=\X+\Y$ je funkce $(f_\X \star g_\Y)(z)$ a platí $\la z
\ra=\mu_1+\mu_2.$\\

\Proof

\begin{itemize}
\item označme $\ZZZ=\X+\Y$ sou\v cet n\'ahodn\'ych veli\v cin

\item pro p\v r\'islu\v snou hustotu pravděpodobnosti veli\v ciny $\ZZZ$ byla ve v\v et\v e \ref{odvozeni_konvoluce} odvozena rovnost $$f_\ZZZ(r)= \int_{-\infty}^\infty f_\X(x)f_\Y(r-x)~\dx,$$ kter\'a reprezentuje prvn\'i z dokazovan\'ych tvrzen\'i

\item zb\'yv\'a proto dok\'azat, \v ze st\v redn\'i hodnotou sou\v ctu n\'ahodn\'ych veli\v cin je sou\v cet st\v redn\'ich hodnot t\v echto veli\v cin

\item použitím Fubiniovy věty, jednoduché substituce a definice střední hodnoty náhodné veličiny dostáváme
%
$$\la z \ra=\int_{\R}z\left(\int_{\R}f(x)g(z-x)~\dx\right)~\dz=\int_\R
f(x)\left(\int_\R z\cdot g(z-x)~\dz\right)~\dx=$$
%
$$=\left|\begin{array}{c} y=z-x\\ \dz=\dy \end{array}\right|= \int_\R
f(x)\left(\int_\R (x+y)\cdot g(y)~\dy\right)~\dx=\int_\R\int_\R
x~f(x)g(y)~\dy~\dx{}+$$
%
$$+\int_\R\int_\R
y~f(x)g(y)~\dy~\dx=\la x\ra\int_{\R} g(y)~\dy+\la y \ra\int_{\R}
f(x)~\dx=\la x \ra+\la y \ra$$

\item t\'im je d\r ukaz proveden

\end{itemize}

\subsection{Věta}
Nech?$\X,$ resp. $\Y$ jsou nezávislé náhodné veličiny maj\'ic\'i absolutn\v e spojit\'e rozd\v elen\'i. Nech\v t jejich hustoty pravděpodobnosti jsou $f_\X(x)\in\LLL_1(\R),$ resp. $g_\Y(y)\in\LLL_1(\R)$ a nav\'ic $\VAR(\X)=\sigma_x^2$ a
$\VAR(\Y)=\sigma_y^2.$ Potom pro rozptyl n\'ahodn\'e veli\v ciny $\ZZZ=\X+\Y$ plat\'i rovnost
%
$$\VAR(\ZZZ)=\sigma_x^2+\sigma_y^2.$$

\Proof

\begin{itemize}

\item hustotou pravd\v epodobnosti n\'ahodn\'e veli\v ciny $\ZZZ=\X+\Y$ je funkce vypo\v cetn\'a jako konvoluce $f(x)~\star~g(x),$  tj. $$h(z)= \int_{\R}f(x)g(z-x)~\dx$$

\item snadno se lze tud\'i\v z p\v resv\v ed\v cit, \v ze plat\'i s\'erie n\'asleduj\'ic\'ich rovnost\'i

\begin{multline*}
\la z^2 \ra=\int_{\R}z^2\left(\int_{\R}f(x)g(z-x)~\dx\right)~\dz=\int_\R
f(x)\left(\int_\R z^2\cdot g(z-x)~\dz\right)~\dx{}=\\
%
=\left|\begin{array}{c} y=z-x\\ \dz=\dy \end{array}\right|= \int_\R
f(x)\left(\int_\R (x+y)^2\cdot g(y)~\dy\right)~\dx=\int_\R\int_\R
x^2~f(x)g(y)~\dy~\dx{}+\\
%
+2\int_\R\int_\R
xy~f(x)g(y)~\dy~\dx+\int_\R\int_\R
y^2~f(x)g(y)~\dy~\dx{}=\\
%
=\la x^2\ra\int_{\R} g(y)~\dy+2\int_{\R^2} xyf(x)g(y)~\dx\dy + \la y^2 \ra\int_{\R}
f(x)~\dx=\la x^2 \ra+2 \la xy \ra+ \la y^2 \ra
\end{multline*}


\item jeliko\v z pro nez\'avisl\'e náhodné veli\v ciny plat\'i, \v ze jejich kovariance je nulov\'a (viz v\v eta \ref{kovariance_nezavislych_velicin}), dost\'av\'ame rovnost $$\VAR(\ZZZ)=\la z^2 \ra - \la z \ra^2 = \la x^2 \ra+2 \la xy \ra+ \la y^2 \ra - \la x \ra^2 - 2 \la x \ra\la y \ra -  \la y\ra^2 =$$ $$=\VAR(\X) + \VAR(\Y) + 2~\COV(\X,\Y)=\VAR(\X) + \VAR(\Y) $$

\item ta ale kompletuje d\r ukaz

\end{itemize}

\subsection{Věta -- \emph{o posunutí v konvoluci}}

Nech\v t jsou d\'any libovoln\'e funkce $f(\x)\in\LLL_1(\Er)$ a $g(\x)\in\LLL_1(\Er)$ a vektor $\b\in\Er.$ Pak plat\'i
%
$$ f(\x+\b) \star g(\x) = f(\x) \star g(\x+\b) = \bigl(f \star g\bigr) (\x+\b).$$

\Proof

\begin{itemize}
\item nen\'i pravd\v epodobn\v e obt\'i\v zn\'e nahl\'ednout, \v ze
%
$$ \bigl(f \star g\bigr) (\x+\b) = \int_{\Er} f(\sss)g(\x+\b-\sss)~\dss = f(\x) \star g(\x+\b) $$
\item d\'ale
%
$$ f(\x+\b) \star g(\x) = \int_{\Er} f(\sss+\b)g(\x-\sss)~\dss = \left|\begin{array}{c} \sss+\b=\rrr\\ \dss=\drr \end{array}\right| = \int_{\Er} f(\rrr)g(\x+\b-\rrr)~\drr= f(\x) \star g(\x+\b)$$
\item p\v ritom existence v\v sech dot\v cen\'ych integr\'al\r u je garantov\'ana v\v etou \ref{existuje_konvo}


\end{itemize}

\subsection{Věta -- \emph{o derivaci konvoluce}}

Nech\v t jsou d\'any libovoln\'e funkce $f(\x)\in\LLL_1(\Er)$ a $g(\x)\in\LLL_1(\Er)$ a vektor $\b\in\Er.$ Nech\v t je $i\in \widehat{r}$ zvoleno libovoln\v e. Nech\v t d\'ale  $\frac{\p f}{\p x_i} \in\LLL_1(\Er)$ a $\frac{\p g}{\p x_i}\in\LLL_1(\Er).$ Pak plat\'i
%
$$ \frac{\p f}{\p x_i} \star g(\x) = f(\x) \star \frac{\p g}{\p x_i} = \frac{\p }{\p x_i}\bigl(f \star g\bigr). $$

\Proof

\begin{itemize}
\item existence v\v sech dot\v cen\'ych integr\'al\r u je op\v et garantov\'ana v\v etou \ref{existuje_konvo}
\item d\'ale
%
$$ \frac{\p f}{\p x_i} \star g(\x) = \int_{\Er} \frac{\p f}{\p s_i}(\sss) g(\x-\sss)~\dss =  \left|\begin{array}{cc} u=g(\sss) & v'=\frac{\p f}{\p s_i}(\sss) \\ u'=\frac{\p g}{\p(x_i-s_i)}\frac{\p(x_i-s_i)}{\p s_i} & v=f(\sss)\end{array}\right|=$$
%
$$=\int_{\E^{r-1}} \Bigl[f(\sss)g(\sss)\Bigr]_{s_i\rightarrow-\infty}^{s_i\rightarrow\infty} \ds_1\ds_2\ldots\ds_{i-1}\ds_{i+1}\ldots\ds_r - \int_{\Er} f(\sss)\frac{\p g}{\p(x_i-s_i)}\frac{\p(x_i-s_i)}{\p s_i}~\dss=$$
%
$$=\int_{\Er} f(\sss)\frac{\p g(\x-\sss)}{\p(x_i-s_i)}~\dss = f(\x) \star \frac{\p g}{\p x_i}$$

\item bylo zde p\v ritom vyu\v zito tzv. \emph{nutn\'e podm\'inky konvergence Lebesgueova integr\'alu,}\index{nutn\'a podm\'inka konvergence Lebesgueova integr\'alu} tedy implikace
%
$$f(\x)\in\LLL_1(\Er)\quad \Longrightarrow \quad \lim_{\|\x\| \rightarrow \infty} f(\x)=0 \quad \Longrightarrow \quad \forall i\in \widehat{r}:~~\lim_{x_i \rightarrow \infty} f(\x)=0.$$

\end{itemize}




\subsection{V\v eta} \label{Konvoluce_hustot}

$f(\x), g(\x):\Er\mapsto\R$ jsou hustoty, pak $\(f\ast g\)(\x)$ je rovn\v e\v z hustotou a v\v zdy existuje.\\

\Proof

\begin{itemize}
\item $f(\x), g(\x)\in \LLL_\onecircled(\Er) \Rightarrow \(f\ast g\)(\x)\in \LLL_1(\Er)$ \textcolor{red}{co znamena 1 v krouzku?} 

\item nez\'apornost:\[\(f\ast g\)(\x)=\int_\Er f(\s)g(\x - \s)~\dss \geq 0 \quad\forall x \in \Er,\]
nebo\v t z definice hustot je integr\' al v\v et\v s\' i nebo roven $0$ a existuje

\item \[
\int_\Er \(f\ast g\)(\x)~\dxx = \int_\Er\int_\Er f(\s)g(\x - \s)~\dss~\dxx = \int_\Er f(\s)\int_\Er g(\x - \s)~\dxx~\dss = 
\]
\[
=\Bigg|\begin{array}{l}\y=\x -\s\\\dyy=\dxx\end{array}\Bigg|=\int_\Er f(\s)\int_\Er g(\y)~\dyy~\dss =1\int_\Er f(\s)~\dss = 1
\]
\end{itemize}

\subsection{Pozn\'amka}
St\v redn\' i hodnota z $r, f(r)$ je $\la r\ra =\int_\R rf(r)~\dr$.

\subsection{V\v eta} \label{Konvoluce_hustot_a_stredni hodnoty}
Nech\v t $f(x), g(x):\R\mapsto\R$ jsou hustoty. Nech\v t $\int_\R xf(x)~\dx = \mu_1$ a $\int_\R xg(x)~\dx = \mu_2$. Pak $\int_\R \(f\ast g\)(x)~\dx = \mu_1 + \mu_2$.\\

\Proof

\begin{itemize}
\item teoretick\' e po\v zadavky ji\v z byly dok\' az\' any v p\v redchoz\' i v\v et\v e
 
\item \[
\int_\R x\(f\ast g\)(x)~\dx = \int_\R x\int_\R f(s)g(x - s)~\ds~\dx = \int_\R f(s)\int_\R xg(x - s)~\dx~\ds = 
\]
\[
=\Bigg|\begin{array}{l}y=x -s\\\dy=\dx\end{array}\Bigg|=\int_\R f(s)\int_\R (y+s)g(y)~\dy~\ds = \int_\R^2 f(s)yg(y)~\ds~\dy + \int_\R^2 f(s)sg(y)~\dy~\ds =
\]
\[
=\Big|\text{V\v eta o separabilit\v e}\Big|=\int_\R f(s)~\ds \int_\R yg(y)~\dy + \int_\R sf(s)~\ds \int_\R g(y)~\dy = \mu_1 + \mu_2
\]
\end{itemize}


\subsection{V\v eta -- \emph{o posunut\' i v konvoluci}}\index{v\v eta o posunut\' i v konvoluci}\label{posunuti_v_konvoluci}

$f(\x), g(\x)\in \LLL_1(\Er), \mumu \in \Er$. Pak plat\' i: $(f\star g)(\x - \mumu)= f(\x)\star g(\x- \mumu) =f(\x - \mumu)\star g(\x)$\\

\subsection{Pozn\'amka}
Zde pou\v z\' iva\' ame afinn\' i transformaci, tud\' i\v z za ka\v zd\' e $\x$ dosad\' ime $\x - \mumu$. Souvislost s p\v redchoz\' i v\v etou je takov\'a, \v ze lze posunout st\v redn\' i hodnotu v p\v r\' ipad\v e, \v ze za $f,g$ zvol\' ime hustoty.


\subsection{V\v eta -- \emph{o derivaci konvoluce}}\index{v\v eta o derivaci konvoluce}\label{derivace_konvoluce}

$f(\x)\in \LLL_1(\Er), g(\x)\in \LLL_1(\Er) \cap \CC_0^1$. Pak plat\' i $\dfrac{\partial}{\partial x_k}(f\star g)= f(\x)\star \dfrac{g}{x_k}(\x)$.\\

\Proof
\begin{itemize}
\item $\dfrac{\partial}{\partial x_k}(f\star g)= \dfrac{\partial}{\partial x_k}\int_\Er f(\s)g(\x - \s)~\dss$
 
\item pou\v zijeme v\v etu o derivaci integr\' alu s parametrem

\item $\dfrac{\mathtt{d}}{\mathtt{d}\alpha}\int_\Er f(\x |\alpha)~\dxx \rightarrow \dfrac{\mathtt{d}}{\mathtt{d}\alpha_k}\int_\Er f(\x |\alpha_1,\alpha_2,\dots,\alpha_n)~\dxx$

\item ov\v e\v rme p\v redpoklady v\v ety:
		\begin{itemize}
		\item v\'yraz v integr\'alu mus\' i konvergovat, co\v z je spln\v eno
		\item m\v e\v ritelnost je spln\v ena, jeliko\v z v\'yraz je z $\LLL_1$
		\item diferencovatelnost, v\'yraz nahrad\'ime integrabiln\'i majorantou: $\Bigg|f(\s)\dfrac{\partial g}{\partial x_k}(\x -\s)\Bigg| \leq K\Big|f(\s)\Big| \in \LLL(\E)$,\\ a vyu\v zijeme vlastnost, \v ze funkce na kompaktu nab\'yv\'a maxima
		\end{itemize}

\item $\int_\Er f(\s)\dfrac{\partial g}{\partial x_k}(\x - \s)~\dss= \Biggl(f\star \dfrac{\partial g}{\partial x_k}\Biggr)(\x)$
\end{itemize}

\subsection{Pozn\'amka}
Pov\v simn\v eme si, \v ze se v\v eta jev\' i na prvn\' i pohled nevyv\'a\v zen\'a, je to z d\r uvodu po\v zadavku na diferencovatelnost pouze pro $g$. Z\'arove\v n si pov\v simn\v eme absence dodatku "pokud lev\'a (prav\'a) strana existuje". U konvoluce pozorujeme tzv. vyhlazovac\'i efekt, kdy pokud je $g(x)$ hladk\'a, pak existuje konvoluce i jej\'i derivace bez ohledu na to, jak nespojit\'a je funkce $f(x)$.

\subsection{P\v r\'iklad}
Spo\v c\'itejme konvoluci dvou Gaussov\'zch funkc\'i. Polo\v zme $f(x)= \dfrac{1}{\sqrt{2\pi}\sigma_1}e^{-\dfrac{(x-\mu_1)^2}{2\sigma_1^2}}$ a $g(x)= \dfrac{1}{\sqrt{2\pi}\sigma_2}e^{-\dfrac{(x-\mu_2)^2}{2\sigma_2^2}}$.
Pak \textcolor{red}{ dopo\v c\'it\'am pozd\v eji.}

 
















