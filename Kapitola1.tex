\chapter{Posloupnosti a \v rady funkc\'i v\'ice prom\v enn\'ych}

%%%%%%%%%%%%%%%%%%%%%%%%%%%%%%%%


\section{Co zpracovat:}


\begin{enumerate}%%%%%%%%%%%%%%%%%%%%%%%%%%%%%%%%%%%%%%%%%%%%%%%%%%%%%%%%%%%%%%%%%%%%%%%%%%%%%%%%%%%%%%%%%%%%%%%%%%
%\item Heavisideova funkce
%\item Prostor s �plnou m�rou $\{\E^r,\lambda(X),\Mla\}$
%\item $G$ bude v�dy zna�it oblast a $J$ bude znamenat kompakt
%\item funkce: $f(\x):\R\mapsto\C$
%\item P�ipomenout $\CC=\CC^0(M),$ $\CC^n(M),$ $\CC_0(M)$ a $\CC^n_0(M)$
%\item $\LLL(G),$ $\LLL^\star(G),$ $\LOC(G)$
%\item p�ipomenout, �e: $$f(x)\in\LLL(E,\mu) \quad \Leftrightarrow \quad|f(x)|\in\LLL(E,\mu)~\wedge f(x)\in \LA_\mu(E).$$
%\item V�ta o ekvivalentn� definici t��dy $\LOC(G)$
%\item snaha o prehilbertovsk� prostor $\CC(\langle a,b\rangle)$
%\item snaha o prehilbertovsk� prostor $\LLL(G)$ a protip��klad $\frac{1}{\sqrt{x}}\in\LLL(0,1),$ ale $\frac{1}{\sqrt{x}}\cdot\frac{1}{\sqrt{x}}=\frac{1}{x}\in\LLL(0,1)$
\item je ale $\CC(\langle a,b\rangle)$ �pln�? (nen�) - \textcolor{red}{zm�nit, okomentovat a vlo�it asi jako pozn�mku za pozn�mku 2.2.12, mo�n� na vhodn�m m�st� zm�nit definici �plnosti (mo�n� u� to n�kde je, te� si nejsem jistej)}
%\item $\LLL_p(G)$
%\item snaha o prehilbertovsk� prostor $\LLL_1(G)$ a protip��klad $\frac{1}{\sqrt{x}}\in\LLL_1(0,1),$ ale $\frac{1}{\sqrt{x}}\cdot\frac{1}{\sqrt{x}}=\frac{1}{x}\in\LLL_1(0,1)$
%\item V�ta $f,g\in\LLL_2(G) \quad \Rightarrow \quad fg^\star\in\LLL_1(G)$
%\item ale ani $\LLL_2(G)$ nen� prehilbertovsk�
%\item Co jsou faktorov� funkce
%\item zna�en� $\LL_2(G)$  a $\LL_2^{(w)}(G):$ v�ha mus� b�t spojit� a kladn� na $G$
\end{enumerate}%%%%%%%%%%%%%%%%%%%%%%%%%%%%%%%%%%%%%%%%%%%%%%%%%%%%%%%%%%%%%%%%%%%%%%%%%%%%%%%%%%%%%%%%%%%%%%%%%%

\subsection{Definice}\label{posloupnost}
Nech\v t $\emptyset \neq M\subset\E^r.$ Potom ka\v zd\'e zobrazen\'i
mno\v ziny $\N$ do mno\v ziny v\v sech funkc\'i definovan\'ych na
$M$ naz\'yv\'ame \emph{posloupnost\'i funkc\'i}\index{posloupnost
funkc�} na $M$. Je-li \v c\'islu $n\in\N$ t\'imto zp\r usobem
p\v ri\v razena funkce $f_{n}(\x)$, zapisujeme funk\v cn\'i
posloupnost
%
\begin{equation}
\label{1.1}
f_{1}(\x),f_{2}(\x), \ldots \hspace{0.5cm} \mathrm{nebo}
\hspace{0.5cm} \(f_n(\x)\)_{n=1}^\infty. 
\end{equation} 
%
P\v rirozen\'e \v c\'islo $n$ p\v ritom naz\'yv\'ame
\emph{indexem}\index{index posloupnosti funkc\'i} a funkci
$f_n(\x)$ $n-$t\'ym \v clenem posloupnosti (\ref{1.1}).

\subsection{Definice}\label{defka_posloupnostxml}
Nech\v t je d\'ana posloupnost funkc\'i (\ref{1.1}) definovan\'a
na nepr\'azdn\'e mno\v zin\v e $M\subset\E^r.$ \v Rekneme, \v ze
posloupnost funkc\'i (\ref{1.1}) \emph{konverguje v bod\v e} $\c\in
M$, jestli\v ze konverguje \v c\'iseln\'a posloupnost
$\(f_n(\c)\)^{\infty}_{n=1},$ tj. existuje-li $\gamma\in\R$
takov\'e, \v ze pro ka\v zd\'e $\ep>0$ existuje p\v rirozen\'e
$n_0$ tak, \v ze pro v\v sechna $n\geq n_0$ plat\'i nerovnost
$\bigl|f_n(\c)-\gamma\bigr|<\ep.$ \v Rekneme, \v ze posloupnost funkc\'i (\ref{1.1})
\emph{konverguje (bodov\v e) na mno\v zin\v e}\index{bodov\'a
konvergence posloupnosti funkc\'i} $N\subset M$, jestli\v ze
konverguje v ka\v zd\'em bod\v e mno\v ziny $N.$


\subsection{Definice}

Nech\v t je d\'ana posloupnost funkc\'i (\ref{1.1}) definovan\'a
na nepr\'azdn\'e mno\v zin\v e $M\subset\E^r.$ Nech\v t pro ka\v
zd\'e $\c\in N,$ kde $N\subset M,$ posloupnost
$\(f_n(\c)\)^{\infty}_{n=1}$ konverguje. Ozna\v cme $f(\c)$ hodnotu
limity posloupnosti $\(f_n(\c)\)^{\infty}_{n=1}.$ T\'imto zp\r
usobem je na mno\v zin\v e $N$ definov\'ana funkce $\x \mapsto
f(\x)$, kterou naz\'yv\'ame \emph{limitou posloupnosti
funkc\'i}\index{limita posloupnosti funkc\'i} (\ref{1.1}) (nebo zkr\'acen\v e \emph{limitn\'i funkc\'i}\index{limitn\'i funkce}) a zna\v
c\'ime $$f(\x)=\limn f_n(\x).$$ \emph{Oborem konvergence} $\OK$
\index{obor konvergence posloupnosti funkc\'i} posloupnosti
(\ref{1.1}) naz\'yv\'ame mno\v zinu v\v sech bod\r u $\c\in M$, ve
kter\'ych tato posloupnost konverguje.

\subsection{Definice}\label{1.8}
Nech\v t (\ref{1.1}) je posloupnost funkc\'i definovan\'ych na
mno\v zin\v e $M\subset\E^r$. \v Rekneme, \v ze tato posloupnost
\emph{stejno\-m\v er\-n\v e konverguje na $M$}\index{stejnom\v
ern\'a konvergence posloupnosti funkc\'i} k funkci $f(\x)$,
jestli\v ze pro v\v sechna $\ep >0$ existuje $n_0$ tak, \v ze pro
v\v sechna $n \geq n_0$ a pro v\v sechna $\x\in M$ plat\'i
nerovnost $\bigl| f_n(\x)-f(\x) \bigr| < \ep.$


\subsection{Pozn\'amka}
Bodovou konvergenci zna\v c\'ime oby\v cejn\v e symbolem $f_n(\x) \rightarrow
f(\x),$ stejnom\v ernou pak  ${f_n}(\x)\rightrightarrows f(\x)$.
Rozd\'il mezi bodovou a stejnom\v ernou konvergenc\'i je dob\v re
patrn\'y z kvantifik\'atorov\'eho z\'apisu definic obou pojm\r u:
%
\begin{itemize}
\item bodov\'a konvergence
\BE(\forall\ep > 0)~(\forall \x\in M)~(\exists n_0\in\N):
~\hspace{1cm}  n\in\N~\wedge~ n \geq n_0 \Rightarrow  \bigl|
f_n(\x)-f(\x) \bigr| < \ep. \EE

\item stejnom\v ern\'a konvergence
\BE(\forall\ep > 0)~(\exists n_0\in\N): ~\hspace{1cm}
n\in\N~\wedge~ n \geq n_0~\wedge~\x\in M \Rightarrow \bigl|
f_n(\x)-f(\x) \bigr| < \ep. \label{1.2} \EE

\end{itemize}
%
Stejnom\v ern\'a konvergence tedy po\v zaduje existenci
"univerz\'aln\'iho" $n_0,$ kter\'e pln\'i svoji roli pro v\v
sechna $\x\in M.$

\subsection{V\v eta -- \emph{Bolzanova-Cauchyova podm\'inka}} \label{Cauchyovo krit\'erium
pro stejnom\v ernou konvergenci posloupnosti funkc\'i}

Posloupnost funkc� (\ref{1.1}) je stejnom\v ern\v e
konvergentn\'i na $M\subset\Er$ pr\'av\v e tehdy, kdy\v z spl\v nuje tzv.
\emph{Bolzanovu-Cauchyovu podm\'inku}\index{Bolzanova-Cauchyova
podm\'inka pro stejnom\v ernou konvergenci posloupnosti funkc\'i}
tvaru \label{1.13}
%
\BE (\forall\ep > 0)~ (\exists n_0\in\N): \hspace{1cm} ~ m,n \geq
n_0~\wedge~ \x\in M \Rightarrow \bigl| f_n(\x)-f_m(\x) \bigr| < \ep.
\label{svetlo} \EE

\Proof

\begin{itemize}
\item Prvn\'i implikace:

\begin{itemize}
\item nech\v t $\(f_n(\x)\)^{\infty}_{n=1}$ stejnom\v ern\v e
konverguje na $M$ k jist\'e funkci $f(x)$

\item pak pro ka\v zd\'e $\ep > 0$ existuje $n_0\in\N$ takov\'e, \v ze pro libovoln\'a $m,n\in\N$ takov\'a, \v ze $m,n \geq
n_0,$ a pro v\v sechna $\x\in M$  plat\'i $$\bigl| f_n(\x)-f(\x)
\bigr|< \frac{\ep}{2}\hspace{0.3cm} \wedge \hspace{0.3cm} \left|
f_m(\x)-f(\x) \right|  < \frac{\ep}{2}$$

\item a tedy $$\bigl|f_n(\x)-f_m(\x)\bigr| \leq \bigl|f_n(\x)-f(\x)\bigr| + \bigl|f_m(\x)-f(\x)\bigr| < \ep$$

\end{itemize}

\item Druh\'a implikace:


\begin{itemize}
\item nech\v t posloupnost funkc\'i spl\v nuje vztah (\ref{svetlo})

\item podle Bolzanovy-Cauchyovy podm\'inky pro \v c\'iseln\'e posloupnosti pos\-loup\-nost (\ref{1.1}) konverguje bodov\v e k jist\'e funkci na mno\v zin\v e $M$
(ozna\v cme ji $f(\x)$)

\item chceme dok\'azat $f_n(\x) \rightrightarrows f(\x)$ na $M$

\item zvolme $\ep>0$ a k \v c\'islu $\frac{\ep}{2}$ vyberme
podle (\ref{svetlo}) $n_0$ tak, aby pro v\v sechna $m,n \geq n_0$
platilo $$\bigl|f_{n}(\x)-f_m(\x)\bigr| < \frac{\ep}{2}$$

\item pro libovoln\'e pevn\v e zvolen\'e  $n \geq  n_0$
a pro $m$ rostouc\'i nade v\v sechny meze pak odsud dostaneme
nerovnost $\bigl|f_{n}(\x)-f(\x)\bigr| \leq \ep/2 < \ep$ platnou pro ka\v zd\'e $\x\in M$

\end{itemize}

\item t\'im je d\r ukaz zkompletov\'an

\end{itemize}


\subsection{V\v eta \emph{-- suprem\'aln\'i krit\'erium}}\index{suprem\'aln\'i krit\'erium}   \label{Seiby}
Nech\v t $f(\x)$ a $f_n(\x)$ pro v\v sechna $n$ jsou funkce
definovan\'e na mno\v zin\v e $M\subset\Er.$ Ozna\v cme
%
$$\sigma_n:=\sup_{\x\in M} \bigl|f_n(\x)-f(\x)\bigr|$$
%
pro ka\v zd\'e $n.$ Pak posloupnost funkc\'i
$\(f_n(\x)\)^{\infty}_{n=1}$ konverguje na mno\v zin\v e $M$
stejnom\v ern\v e k funkci $f(\x)$ pr\'av\v e tehdy, kdy\v z $\lim_{n \rightarrow \infty} \sigma_n =0.$\\

\Proof

\begin{itemize}
\item pro v\v sechna $\x\in M$ a v\v sechna $n\in\N$  z\v rejm\v e plat\'i nerovnost $\bigl|f_n(\x)-f(\x)\bigr|
\leq \sigma_n$
\item Prvn\'i implikace:
\begin{itemize}
\item p\v redpokl\'adejme, \v ze $\lim_{n \rightarrow \infty} \sigma_n =0$
\item z definice limity \v c\'iseln\'e posloupnosti $\(\sigma_n\)_{n=1}^\infty$ plyne, \v ze pro libovoln\'e $\ep>0$ existuje $n_0$ takov\'e, \v ze $|\sigma_n|=\sigma_n <\ep$ pro v\v sechna $n\geq n_0$
\item to zna\v c\'i (jak vypl\'yv\'a z definice suprema), \v ze pro v\v sechna $n \geq n_0$ a v\v sechna $\x\in
M$ plat\'i tak\'e $\bigl|f_n(\x)-f(\x)\bigr| < \ep,$ a tedy $ f_n(\x) \rightrightarrows f(\x)$ na $M$
\end{itemize}
\item Druh\'a implikace:
\begin{itemize}
\item p\v redpokl\'adejme, \v ze $f_n(\x) \rightrightarrows f(\x)$ na $M$
\item zvolme libovoln\'e $\ep>0,$ k n\v emu\v z jist\v e existuje $n_0$ takov\'e, \v ze pro v\v sechna $n\geq
n_0$ a v\v sechna $\x \in M$ plat\'i nerovnost $\bigl|f_n(\x)-f(\x)\bigr| < \ep/2$
\item odtud a z vlastnost\'i suprema plyne, \v ze pro $n\geq
n_0$ plat\'i $\sigma_n \leq \ep/2 < \ep,$ a tedy $\limn \sigma_n =0$
\end{itemize}
\end{itemize}


\subsection{Definice}
Nech\v t je d\'ana posloupnost funkc\'i (\ref{1.1}) definovan\'a
na nepr\'azdn\'e mno\v zin\v e $M\subset\E^r$. Potom nekone\v cn\'y
sou\v cet $$f_1(\x)+f_2(\x)+\ldots+f_n(\x)+\ldots$$
%
naz\'yv\'ame \emph{\v radou funkc\'i}\index{\v rada funkc\'i} na
$M$ a zna\v c\'ime symbolem \BE \sum^{\infty}_{n=1}f_n(\x).
\label{Tyky} \EE


\subsection{Definice}
Nech\v t je d\'ana funk\v cn\'i \v rada (\ref{Tyky}) definovan\'a
na mno\v zin\v e $M.$ Funkci $s_n(\x)=\sum^{n}_{k=1}f_k(\x)$ pro
$n\in\N$ a $\x\in M$ budeme naz\'yvat \emph{$n-$t\'ym \v c\'aste\v
cn\'ym sou\v ctem}\index{\v c\'aste\v cn\'y sou\v cet \v rady
funkc\'i} \v rady (\ref{Tyky}) a posloupnost
$\bigl(s_n(\x)\bigr)_{n=1}^\infty$ pak \emph{posloupnost\'i \v
c\'aste\v cn\'ych sou\v ct\r u}\index{posloupnost \v c\'aste\v
cn\'ych sou\v ct\r u} dan\'e \v rady.

\subsection{Definice}

Nech\v t je d\'ana funk\v cn\'i \v rada (\ref{Tyky}) definovan\'a
na mno\v zin\v e $M.$ Nech\v t $\bigl(s_n(\x)\bigr)_{n=1}^\infty$
je p\v r\'islu\v sn\'a posloupnost \v c\'aste\v cn\'ych sou\v ct\r
u. \v Rekneme, \v ze \v rada (\ref{Tyky}) \emph{konverguje
v~~bod\v e} $\c\in M$, jestli\v ze konverguje \v c\'iseln\'a
posloupnost $\(s_n(\c)\)^{\infty}_{n=1}.$ \v Rekneme, \v ze \v rada
(\ref{Tyky}) \emph{konverguje (bodov\v e)}\index{bodov\'a
konvergence \v rady funkc\'i} na mno\v zin\v e $N\subset M$,
jestli\v ze konverguje v ka\v zd\'em bod\v e mno\v ziny $N.$
Vlastn\'i limitu $$s(\x):=\limn s_n(\x)$$ posloupnosti \v c\'aste\v
cn\'ych sou\v ct\r u pak naz\'yv\'ame \emph{sou\v ctem \v
rady}\index{sou\v cet \v rady funkc\'i} (\ref{Tyky}) a zapisujeme
%
\BE s(\x)=\sum^{\infty}_{n=1}f_n(\x). \label{2.1} \EE
%
Defini\v cn\'i obor $\Dom(s),$ tj. mno\v zinu v\v sech $\c\in M,$
pro n\v e\v z posloupnost $\(s_n(\c)\)^{\infty}_{n=1}$ konverguje,
budeme d\'ale naz\'yvat \emph{oborem konvergence \v
rady}\index{obor konvergence \v rady funkc\'i} (\ref{Tyky}) a zna\v
cit symbolem $\OK$.


\subsection{Definice}
\v Rekneme, \v ze \v rada funkc\'i $\sum_{n=1}^\infty f_n(\x)$ konverguje na mno\v zin\v e
$M\subset\E^r$ \emph{stejnom\v ern\v e}\index{stejnom\v ern\'a
konvergence \v rady funkc\'i} ke sv\'emu sou\v ctu $s(\x)$ a ozna\v
c\'ime $\sum^{\infty}_{n=1}f_n(\x) \stackrel{M}{\equiv} s(\x),$ jestli\v ze
posloupnost jej\'ich \v c\'aste\v cn\'ych sou\v ct\r u konverguje
na $M$ stejnom\v ern\v e k funkci $s(\x).$

\subsection{V\v eta \emph{-- Bolzanova-Cauchyova podm\'inka}}\label{2.7}
\index{Bolzanova-Cauchyova podm\'inka pro stejnom\v ernou
konvergenci \v rady funkc\'i}

\v Rada funkc\'i (\ref{Tyky}) konverguje na mno\v zin\v e
$M\subset\Er$ stejnom\v ern\v e pr\'av\v e tehdy, kdy\v z pro ka\v
zd\'e $\ep>0$ existuje index $n_0\in\N$ takov\'y, \v ze pro
jak\'ekoli dva indexy $m,n\in\N$ takov\'e, \v ze $m \geq n \geq
n_0$ a pro jak\'ekoliv $\x\in M$ je spln\v ena nerovnost
$$\bigl| f_n(\x)+f_{n+1}(\x)+\ldots+f_m(\x) \bigr| < \ep.$$

\Proof

\begin{itemize}

\item tvrzen\'i t\'eto v\v ety bezprost\v redn\v e plyne z v\v ety \ref{1.13}
\item ozna\v c\'ime-li toti\v z $\bigl(s_n(\x)\bigr)_{n=1}^\infty$
p\v r\'islu\v snou posloupnost \v c\'aste\v cn\'ych sou\v ct\r u,
z\'isk\'av\'ame rovnosti $$s_{n-1}(\x)=\sum_{k=1}^{n-1} f_k(\x),
\hspace{1cm} s_m(\x)=\sum_{k=1}^m f_k(\x)$$
\item podle v\v ety \ref{1.13} (v nepatrn\'e obm\v en\v e) konverguje posloupnost
$\bigl(s_n(\x)\bigr)_{n=1}^\infty$ na $M$ stejnom\v ern\v e
pr\'av\v e tehdy, kdy\v z pro ka\v zd\'e $\ep>0$ existuje index
$n_0\in\N$ takov\'y, \v ze pro jak\'ekoli dva indexy $m,n\in\N$
takov\'e, \v ze $m \geq n \geq n_0$ a pro jak\'ekoliv $\x\in M$ je
spln\v ena nerovnost $\bigl| s_m(\x)-s_{n-1}(\x) \bigr| < \ep$

\item z t\'eto nerovnosti ov\v sem vypl\'yv\'a, \v ze $$\left| \sum_{k=1}^m f_k(\x)-\sum_{k=1}^{n-1} f_k(\x) \right| = \bigl| f_n(\x)+f_{n+1}(\x)+\ldots+f_m(\x) \bigr| < \ep$$

\end{itemize}

\subsection{Definice}
\v Rekneme, \v ze \v rada funkc\'i $\sum^{\infty}_{n=1}f_n(\x)$
konverguje na mno\v zin\v e $M\subset\Er$ \emph{regul\'arn\v
e}\index{regul\'arn\'i konvergence \v rady funkc\'i}, jestli\v ze
\v rada $\sum^{\infty}_{n=1} \bigl|f_n(\x)\bigr|$ konverguje na $M$
stejnom\v ern\v e.

\subsection{V\v eta \emph{-- nutn\'a podm\'inka stejnom\v ern\'e
konvergence}}\index{nutn\'a podm\'inka stejnom\v ern\'e
konvergence}

Jestli\v ze \v rada funkc\'i $\sum^{\infty}_{n=1}f_n(\x)$
konverguje na mno\v zin\v e $M\subset\Er$ stejnom\v ern\v e, potom
posloupnost funkc\'i $\bigl(f_n(\x)\bigr)_{n=1}^\infty$ konverguje
na t\'eto mno\v zin\v e stej\-no\-m\v er\-n\v e k nulov\'e
funkci.\\

\Proof

\begin{itemize}

\item z p\v redpoklad\r u v\v ety plyne, \v ze $$(\forall\ep > 0)(\exists n_0\in \N)(\forall m,n\in\N)\bigl(m \geq n
\geq n_0\bigr)(\forall \x\in M ):~\hspace{0.3cm}\bigl|
f_n(\x)+f_{n+1}(\x)+\ldots+f_m(\x) \bigr| < \ep$$

\item jeliko\v z toto tvrzen\'i plat\'i pro jak\'akoli $m,n\in\N$ takov\'a, \v ze $m \geq n
\geq n_0,$ plat\'i tak\'e p\v ri speci\'aln\'i volb\v e $m=n$

\item pak ale $$(\forall\ep > 0)(\exists n_0\in \N)(\forall n\in\N)\bigl(n
\geq n_0\bigr)(\forall \x\in M ):~\hspace{0.3cm}\bigl| f_n(\x)
\bigr|=\bigl| f_n(\x)-o(\x) \bigr| < \ep$$

\item tento v\'yrok je ale ekvivalentn\'i tvrzen\'i, \v ze
posloupnost funkc\'i $\bigl(f_n(\x)\bigr)_{n=1}^\infty$ konverguje
na mno\v zin\v e $M$ stej\-no\-m\v er\-n\v e k nulov\'e funkci

\end{itemize}

\subsection{Definice}
Nech\v t jsou d\'any funk\v cn\'i \v rady $\sum_{n=1}^\infty
f_n(\x)$ a $\sum_{n=1}^\infty g_n(\x)$ definovan\'e na mno\v zin\v e
$M.$ Nech\v t existuje $n_0\in\N$ tak, \v ze pro v\v sechna $n\geq
n_0$ a v\v sechna $\x\in M$ plat\'i $\bigl|f_n(\x)\bigr| \leq
g_n(\x).$ Pak \v radu $\sum_{n=1}^\infty g_n(\x)$ naz\'yv\'ame \v
radou  \emph{majorantn\'i}\index{majorantn\'i \v rada funkc\'i} k
\v rad\v e $\sum_{n=1}^\infty f_n(\x).$

\subsection{V\v eta \emph{-- srovn\'avac\'i krit\'erium}}\index{srovn\'avac\'i krit\'erium pro stejnom\v ernou konvergenci}

Nech\v t \v rada $\sum^{\infty}_{n=1}g_n(\x)$ je na mno\v zin\v e
$M\subset\Er$ majorantn\'i k~\v rad\v e $\sum^{\infty}_{n=1}f_n(\x)$ a nech\v
t \v rada $\sum^{\infty}_{n=1}g_n(\x)$ je stejno\-m\v er\-n\v e
konvergentn\'i na $M$. Pak jsou \v rady $\sum^{\infty}_{n=1}f_n(\x)$ a
$\sum^{\infty}_{n=1}\left|f_n(\x)\right|$ stejnom\v ern\v e
konvergentn\'i na $M,$ tj. \v rada $\sum^{\infty}_{n=1}f_n(\x)$
konverguje na $M$ regul\'arn\v e.\\

\Proof

\begin{itemize}

\item u\v zijeme Bolzanovu-Cauchyovu podm\'inku \ref{2.7}

\item z p\v redpokladu v\'ime, \v ze \v rada $\sum_{n=1}^\infty g_n(\x)$ stejnom\v ern\v e konverguje na $M,$ tedy pro jak\'ekoli $\ep>0$ existuje $n_0$ takov\'e, \v ze pro v\v sechna p\v rirozen\'a $m \geq n \geq n_0$ a pro v\v sechna $\x \in M$ plat\'i
%
$$0 \leq g_n(\x)+g_{n+1}(\x)+\ldots+g_m(\x)<\ep$$

\item d\'ale v\'ime, \v ze existuje $m_0$ tak, \v ze pro v\v sechna $x\in M$ a v\v sechny indexy $n \geq m_0$ plat\'i $|f_n(\x)| \leq g_n(\x)$

\item pro zvolen\'e $\ep$ a v\v sechna $n\geq \max \{n_0,m_0\}$ pak plat\'i $$
|f_n(\x)+f_{n+1}(\x)+\ldots+f_m(\x)|\leq|f_n(\x)|+|f_{n+1}(\x)|+\ldots+|f_m(\x)|\leq g_n(\x)+g_{n+1}(\x)+\ldots+g_m(\x)<\ep$$

\item to dokazuje ob\v e tvrzen\'i v\v ety

\end{itemize}

\subsection{D\r usledek}
Konverguje-li \v rada na mno\v zin\v e $M$ regul\'arn\v e,
konverguje na $M$ tak\'e stejnom\v ern\v e.


\subsection{V\v eta -- \emph{Weierstrassovo krit\'erium}} \label{W-krit}
\index{Weierstrassovo krit\'erium} Nech\v t
$\sum^{\infty}_{n=1}a_n$ je konvergentn\'i \v c\'iseln\'a \v rada,
$f_n(\x)$ jsou funkce a pro v\v sechna $\x\in M\subset\Er$ a v\v sechna $n\in
\N\setminus \widehat{n_0}$ je $\left|f_n(\x)\right|\leq a_n.$ Pak \v rady
$\sum^{\infty}_{n=1}f_n(\x)$ a
$\sum^{\infty}_{n=1}\left|f_n(\x)\right|$ stejnom\v ern\v e
konverguj\'i na $M,$ tj. \v rada $\sum^{\infty}_{n=1}f_n(\x)$
konverguje na $M$ regul\'arn\v e.\\

\Proof

\begin{itemize}

\item v p\v redchoz\'i v\v et\v e polo\v z\'ime $g_n(\x):=a_n$ pro
v\v sechna $\x\in M$ a uv\v edom\'ime si, \v ze pojmy bodov\'e a
stejnom\v ern\'e konvergence u \v rady konstantn\'ich funkc\'i
spl\'yvaj\'i

\end{itemize}

