\chapter{Funkcion\'aln\'i Hilbertovy prostory}

\section{V\'ychoz\'i pojmy}

\subsection{Značení}

$\CC^n(M)$ je třída všech funkcí, které mají na množině $M$ spojité derivace až do řádu $n$, přičemž $\CC(M)=\CC^0(M)$. Nachází-li se index nula dole $\CC^n_0(M)$, pak $M$ je kompakt. Symbol $\CC^n_0$ značí všechny funkce třídy $\CC^n(\Er)$, které mají libovolný, ale kompatní nosič. $\LLL(G)$ je třída Lebesgueovsky integrovatelných funkcí na množině G. Třída funkcí majících Lebesgueoův integrál na G se značí $\LLL^\star(G).$ Třídu Lebesgueovsky lokálně integrabilních funkcí značíme $\LOC(G)$ a definujeme ji v následujícím textu.

\subsection{Úmluva}

Symbol $G$ bude nadále reprezentovat $r-$dimenzionální \emph{oblast}, tj. otevřenou a souvislou podmnožinu množiny $\Er.$ 
Dále symbol $J$ bude označovat \emph{kompakt}, tj. uzavřenou a omezenou podmnožinu množiny $\Er.$
Funkcí budeme rozumět zobrazení $f(\x):\Er\mapsto\C.$


\subsection{Úmluva}

V celém následujícím textu budeme předpokládat, že je zadána klasická a úplná Lebesgueova míra $\lambda(X):\Mlambda\mapsto \R^\star$ generovaná ve všech dimenzích klasickou vytvořující $\phi(x)=x.$ Tudíž soustava $\Mlambda$ všech $\lambda-$měřitelných podmnožin množiny $\Er$ je $\sigma-$algebrou a $\lambda(X)$ je na ní $\sigma-$aditivní mírou. Systém $\bigl\{\Er,\Mlambda,\lambda(X)\bigr\}$ je tedy pro nás nyní výchozím prostorem s úplnou mírou.



\subsection{Definice}

Nechť $r\in\N$ a $\mumu\in\R^r.$ \emph{Heavisideovou}\index{Heavisideova funkce} \texttt{[hevisajdovou]} funkcí budeme rozumět
funkci $\Theta(\x):~\E^r \mapsto \{0,1\}$ definovanou předpisem
%
\BE \Theta(\x):=\left\{
\begin{array}{cll}
    1 & \ldots & x_1 >0 ~\wedge~ x_2 >0 ~\wedge~ \ldots ~\wedge~ x_r >0  \\
    0 & \ldots & x_1 \leq 0 ~\vee~ x_2 \leq 0 ~\vee~ \ldots ~\vee~ x_r \leq 0.  \\
\end{array}
\right. \label{Heaviside}\EE
%
\emph{Centrovanou Heavisideovou}\index{centrovaná Heavisideova funkce} funkcí budeme rozumět
funkci $\Theta_\mumu(\x):~\E^r \mapsto \{0,1\}$ definovanou předpisem
%
\BE \Theta_\mumu(\x):=\left\{
\begin{array}{cll}
    1 & \ldots & x_1 > \mu_1 ~\wedge~ x_2 > \mu_2 ~\wedge~ \ldots ~\wedge~ x_r > \mu_r  \\
    0 & \ldots & x_1 \leq \mu_1 ~\vee~ x_2 \leq \mu_2 ~\vee~ \ldots ~\vee~ x_r \leq \mu_r.  \\
\end{array}
\right. \label{Heaviside-centered}\EE


\subsection{Pozn\'amka}

Funkce $f(\x)$ je, podle věty 5.3.45 a důsledku 5.3.46 v \cite{Teorie_Míry_Krbálek}, na G Lebesgueovsky integrabilní právě tehdy, když je $\lambda$-měřitelná a její absolutní hodnota je Lebesgueovsky integrabilní.

$$f(\x)\in\LLL(G,\mu) \quad \Leftrightarrow \quad|f(x)|\in\LLL(G,\mu)~\wedge f(x)\in \LA_\mu(G).$$
Budeme-li tedy mluvit o měřitelných funkcích, tak platí, že

$$f(x)\in\LLL(G,\mu) \quad \Leftrightarrow \quad |f(x)|\in\LLL(G,\mu)$$


\subsection{Definice}

Nechť je dána funkce $f(\x):G\mapsto\R.$ Řekneme, že funkce
$f(\x)$ je \emph{lokálně integrabilní}\index{lokálně integrabilní funkce} na $G$ a označíme
symbolem $f(\x)\in \LOC \bigl(G,\mu(X)\bigr)$ nebo zkráceně
$f(\x)\in \LOC(G),$ jestliže pro každý bod $\ccc \in G$ existuje $\ep>0$ tak, že $f(\x)\in \LLL\bigl(\U_\ep(\ccc)\bigr),$ tj.
%
$$\int_{\U_\ep(\ccc)} f(\x)~\dmu(\x)\in\R.$$

\subsection{Věta}

Nechť $G$ je oblast v $\Er.$ Funkce $f(\x):G\mapsto\R$ je lok\'aln\v e integrabiln\'i na $G$ pr\'av\v e tehdy, kdy\v z pro ka\v zdou kompaktn\'i
mno\v zinu $J\subset G$ plat\'i, \v ze
%
$$\int_{J} f(\x)~\dmu(\x)\in\R.$$

\Proof

\begin{itemize}
\item dokážeme nejprve, že pokud pro každou kompaktní množinu $J\subset G$ platí, že integrál $\int_{J} f(\x)~\dmu(\x)$ konverguje, pak je  $f(\x)$ je lokálně integrabilní na $G$
\item zvolme tedy libovolně bod $\c\in G$
\item jelikož $G$ je otevřená, jistě existuje $\ep>0$ tak, že $K=\overline{\U_\ep(\c)},$ $K\subset G,$ $K$ je kompakt  a $\c\in \U_\ep(\c)$
\item integrál $\int_{K} f(\x)~\dmu(\x)$ ale existuje z předpokladu
\item $\bd(K)$ je $\mu-$nulová množina, nebo?se jedná o pláš?r-rozměrné koule, a z teorie Lebesgueova integrálu tudíž platí, že $\int_{K} f(\x)~\dmu(\x)=\int_{\U_\ep(\c)} f(\x)~\dmu(\x)$, a navíc jsme $\c$ volili libovolně. 
\item pro důkaz obr\'acen\'e implikace p\v redpokl\'adejme, \v ze  $f(\x)$ je lok\'aln\v e integrabiln\'i na $G$
\item zvolme $K$ jako libovolnou kompaktn\'i množinu, která je podmno\v zinou oblasti $G$
\item podle teorie m\'iry jist\v e $K\in\Mmu $, nebo\v t $\Er\setminusK\in\SSS_r\subset\Mmu $, a $\Mmu$ je $\sigma$-algebra \textcolor{red}{zde je nedefinovany prikaz \textbackslash setminusK, co ma znamenat?}
\item Borelova v\v eta ale \v r\'ik\'a, že z ka\v zd\'eho otev\v ren\'eho pokryt\'i kompaktn\'i mno\v ziny lze vybrat pokryt\'i kone\v cn\'e, tj. existuje soustava oblast\'i $\{G_k: k\in \widehat{n}\}$ tak, \v ze $\cup_{k=1}^n G_k \supset K$ a $G_k=\U_{\ep}(\x_k)$ pro jist\'e body $\x_k\in K$
\item v\v sechny integr\'aly $\int_{\U_\ep(\x_k)} f(\x)~\dmu(\x)$ ale existuj\'i z p\v redpokladu t\'eto implikace
\item d\'ale tak\'e existuj\'i (jak v\'ime z teorie Lebesgueova integr\'alu v\v sechny integr\'aly) $\int_{\U_\ep(\x_k)\cap \U_\ep(\x_\ell)} f(\x)~\dmu(\x)$ pro $k,\ell\in \widehat{n}$
\item existuj\'i rovn\v e\v z integr\'aly $\int_{\U_\ep(\x_k) \cap K} f(\x)~\dmu(\x),$ co\v z spole\v cn\v e garantuje existenci integrálu $\int_{K} f(\x)~\dmu(\x)$
\item tímto je důkaz dokončen
\end{itemize}

\section{Prehilbertovsk\'e prostory funkc\'i}

V této sekci se pokusíme rozhodnout jestli z vybraných vektorových prostorů funkcí lze vytvořit prehilbertovské prostory funkcí, tj. vektorové prostory se skalárním součinem. Připomeňme si definici skalárního součinu.


\subsection{Definice}

Nechť $\V$ je libovoln\'y vektorov\'y prostor nad t\v elesem
$\C.$ Zobrazen\'i $\la . | . \ra: \V\times \V\mapsto\C$ nazveme
\emph{skal\'arn\'im sou\v cinem}\index{skal\'arn\'i sou\v cin},
jestli\v ze spl\v nuje tzv. \emph{axiomy skal\'arn\'iho sou\v
cinu}:\index{axiomy skal\'arn\'iho sou\v cinu}
\begin{itemize}
    \item \emph{lev\'a linearita:}\index{lev\'a linearita skal\'arn\'iho
sou\v cinu} pro v\v sechna $f(\x) ,g(\x),h(\x)\in \V$ a ka\v zd\'e $\alpha
\in \C$ plat\'i $\la\alpha f+g|h\ra=\alpha\la f |h\ra+\la g
|h\ra$
    \item \emph{hermiticita:}\index{hermiticita skal\'arn\'iho
sou\v cinu} pro v\v sechna $f(\x) ,g(\x)\in \V$ plat\'i $\la f
|g\ra=\la g|f \ra^\star$
    \item \emph{pozitivn\'i definitnost:}\index{pozitivn\'i definitnost skal\'arn\'iho
sou\v cinu} pro v\v sechna $f(\x) \in \V$ plat\'i $\la f|f \ra\geq
0$ a nav\'ic $\la f|f \ra=0$ pr\'av\v e tehdy, kdy\v z $f(\x) =o(\x).$
\end{itemize}
%
Dvojici $\bigl\{\V,\la . | . \ra\bigr\}$ naz\'yv\'ame
\emph{prehilbertovsk\'ym prostorem.}\index{prehilbertovsk\'y
prostor}



\subsection{Definice}
Nechť $\V$ je vektorov\'y prostor funkc\'i nad t\v elesem $\C.$
Zobrazen\'i $\left\|~.~\right\|: \V\mapsto\R$ nazveme
\emph{normou}\index{norma}, jestli\v ze spl\v nuje tzv.
\emph{axiomy normy}:\index{axiomy normy}
\begin{itemize}
    \item \emph{nulovost:}\index{nulovost normy} $\left\|f \right\|=0$ pr\'av\v e tehdy, kdy\v z $f(\x) =o(\x)$
    \item \emph{troj\'uheln\'ikov\'a nerovnost:}\index{troj\'uheln\'ikov\'a nerovnost normy} pro v\v sechna $f(\x) ,g(\x)\in \V$ plat\'i: $\left\|f +g\right\|\leq\left\|f \right\|+\left\|g\right\|$
    \item \emph{homogenita:}\index{homogenita normy} pro v\v sechna $f(\x) \in \V$ a ka\v zd\'e $\lambda\in\C$ plat\'i: $\left\|\lambda ~ f \right\|= \left|\lambda\right| \left\|f \right\|.$
\end{itemize}
%
Dvojici $\bigl\{\V,\left\|.\right\|\bigr\}$ naz\'yv\'ame
\emph{normovan\'ym prostorem}.\index{normovan\'y prostor}


\subsection{P\v r\'iklad}

Uk\'a\v zeme, \v ze pro libovolnou funkci $f(\x)\in \V$ z
normovan\'eho prostoru $\V$ s normou $\left\|~.~\right\|$ plat\'i
nerovnost $\left\|f\right\|\geq 0.$ Nejprve snadno prok\'a\v
zeme, \v ze norma opa\v cn\'eho vektoru je stejn\'a jako norma
vektoru p\r uvodn\'iho. Polo\v zme $\lambda=-1.$ Pak z axiomu
homogenity plyne $\left\|-f\right\|= \left|-1\right| \left\|f\right\|=\left\|f\right\|.$ D\'ale pak v troj\'uheln\'ikov\'e nerovnosti polo\v zme $g(\x):=-f(\x).$ Pak $$0=\|o(\x)\|=\left\|f(\x)+\bigl(-f(\x)\bigr)\right\| \leq \bigl\|\f(\x)\bigr\|+\bigl\|-f(\x)\bigr\|=2\bigl\|\f(\x)\bigr\|,$$ odkud je ji\v z patrno, \v ze $\|f\|\geq 0.$


\subsection{V\v eta}
Nech\v t $\la . | . \ra$ je skal\'arn\'i sou\v cin definovan\'y na
vektorov\'em prostoru $\V$ nad t\v elesem $\C.$ Pak zobrazen\'i
$\mathbbm{n}(f)$ definovan\'e p\v redpisem
%
\BE \mathbbm{n}(f):=\sqrt{\la f|f\ra} \label{ngss} \EE
%
je normou na $\V.$\\

\Proof

\begin{itemize}
\item ov\v e\v r\'ime axiomy normy
\item axiom nulovosti:

\begin{itemize}
\item je-li $f(\x)=0,$ pak $\mathbbm{n}^2(0):=\la o,o\ra=0$
\item je-li $\mathbbm{n}(f)=0,$ pak tedy $\la f,f \ra=0,$ ale podle
axiomu pozitivn\'i definitnosti skal\'arn\'iho sou\v cinu toto m\r
u\v ze nastat pouze tehdy, je-li $f(\x)=o(\x)$
\item t\'im je ekvivalence po\v zadovan\'a v axiomu nulovosti
normy prok\'az\'ana
\end{itemize}

\item axiom troj\'uheln\'ikov\'e nerovnosti:

\begin{itemize}
\item provedeme n\'asleduj\'ic\'i s\'erii \'uprav
$$\mathbbm{n}^2\bigl(f+g\bigr)=\la f+g|f+g\ra=\la f|f\ra+\la f|g \ra+\la g|f\ra+\la g|g \ra=$$
%
$$=2~\mathrm{Re}\bigl(\la f|g \ra\bigr)+\la f|f \ra+\la g|g \ra \leq 2\bigl|\la f|g \ra\bigr|+\mathbbm{n}^2(f)+\mathbbm{n}^2(g)$$

\item u\v zijeme-li nyn\'i Schwarzovy-Cauchyovy-Bunjakovsk\'eho nerovnosti
(viz \cite{Krbalek_MAB3}), dost\'av\'ame
$$\mathbbm{n}^2\bigl(f+g\bigr) \leq
2~\mathbbm{n}(f)\mathbbm{n}(g)+\mathbbm{n}^2(f)+\mathbbm{n}^2(g)=\bigl(\mathbbm{n}(f)+\mathbbm{n}(g)\bigr)^2$$

\item t\'im je dok\'az\'ano, \v ze $\mathbbm{n}\bigl(f+g\bigr)\leq \mathbbm{n}(f)+\mathbbm{n}(g)$

\end{itemize}


\item axiom homogenity:

\begin{itemize}
\item nech\v t tedy $\lambda\in\C$ je zvoleno libovoln\v e
\item pak snadno $\mathbbm{n}\bigl(\lambda f\bigr):=\sqrt{\la\lambda f|\lambda f\ra}=\sqrt{\lambda\lambda^\star}
\sqrt{\la f|f \ra}=\sqrt{|\lambda|^2}~\mathbbm{n}(f)=|\lambda|~\mathbbm{n}(f)$

\end{itemize}

\item t\'im je prok\'az\'ano, \v ze zobrazen\'i $\mathbbm{n}(f)$ je normou na $\V$

\end{itemize}


\subsection{Definice}\label{jiste}

Nech\v t $\la . | . \ra$ je skal\'arn\'i sou\v cin definovan\'y na
vektorov\'em prostoru $\V$ nad t\v elesem $\C.$ Pak zobrazen\'i
$\mathbbm{n}(f)$ definovan\'e vztahem (\ref{ngss}) naz\'yv\'ame
\emph{normou generovanou skal\'arn\'im sou\v cinem.}\index{norma
generovan\'a skal\'arn\'im sou\v cinem}


\subsection{V\v eta}\label{love_so_beautiful}

Nech\v t je d\'an vektorov\'y prostor $\V$ nad t\v elesem $\C$ a
skal\'arn\'i sou\v cin $\la . | . \ra.$ Nech\v t $\|.\|$ je norma
generovan\'a t\'imto skal\'arn\'im sou\v cinem. Nech\v t je d\'ana posloupnost funkc\'i $(f_n(\x))_{n=1}^\infty$ z prostoru $\V,$ pro n\'i\v z existuje funkce $f(\x)\in\V$ tak, \v ze plat\'i n\'asleduj\'ic\'i implikace:
%
$$(\forall \ep>0)(\exists n_0\in\N): \quad n>n_0 \quad \Longrightarrow  \quad \bigl\|f_n(\x)-f(\x)\bigr\|<\ep.$$
%
Nech\v t je funkce $g(\x)\in\V$ zvolena libovoln\v e. Pak plat\'i
%
$$\limn \la f_n|g \ra = \la f|g \ra, \quad \limn \la g|f_n \ra = \la g|f \ra.$$

\Proof

\begin{itemize}
\item snadno nahl\'edneme, \v ze pro $g(\x)=o(\x)$ plat\'i citovan\'a rovnost trivi\'aln\v e
\item uva\v zujme tedy nyn\'i pouze ty funkce, kter\'e nejsou nulov\'e, tedy ty, pro n\v e\v z $\|g(\x)\|\neq 0$
\item chceme dok\'azat, \v ze \v c\'iseln\'a posloupnost $(\gamma_n)_{n=1}^\infty,$ kde $\gamma_n:= \la f_n|g \ra$ konverguje k \v c\'islu $\gamma:= \la f|g \ra$
\item je tedy t\v reba prok\'azat, \v ze pro ka\v zd\'e $\ep>0$ existuje $m_0\in\N$ tak, \v ze pro v\v sechny indexy $m>m_0$ plat\'i nerovnost $|\gamma_m-\gamma|<\ep$
\item z p\v redpokladu
%
$$(\forall \ep>0)(\exists n_0\in\N): \quad n>n_0 \quad \Longrightarrow  \quad \bigl\|f_n(\x)-f(\x)\bigr\|<\frac{\ep}{\|g\|},$$
%
z axiom\r u skal\'arn\'iho sou\v cinu a z Schwarzovy-Cauchyovy-Bunjakovsk\'eho nerovnosti ale vypl\'yv\'a, \v ze
%
$$ |\gamma_m-\gamma| = \bigl|\la f_m|g \ra-\la f|g \ra\bigr| = \bigl|\la f_m-f|g  \ra\bigr| \leq \|f_m-f\|\cdot \|g\| < \frac{\ep}{\|g\|}\|g\| =\ep$$
\item posta\v c\'i tedy volit $m_0:=n_0$
\item tvrzen\'i $\limn \la g|f_n \ra = \la g|f \ra$ lze dok\'azat zcela analogicky
\end{itemize}




\subsection{Lemma}

Nech\v t $a\in\R$ a $b\in(a,\infty).$ Nech\v t $\CC(\langle a,b \rangle)$ je vektorov\'y prostor v\v sech
funkc\'i $f(x):~\R\mapsto\C$ spojit\'ych na intervalu $\la a,b
\ra$ zaveden\'y nad t\v elesem $\C.$ Nech\v t je d\'ana funkce $w(x)\in\CC(\langle a,b \rangle)$ kladná na $\la a,b \ra.$ Pak formule
%
%
\BE \bla f(x)|g(x)\bra_{w}:=\int_a^b f(x)g^\star(x)w(x)~\dx 
\label{skalarni_soucin}\EE
%
spl\v nuje axiomy skal\'arn\'iho sou\v cinu na $\CC(\langle a,b \rangle).$


\subsection{Lemma}

Nech\v t $a\in\R$ (nebo $a=-\infty$) a $b\in(a,\infty)$ (nebo $b=+\infty$). Nech\v t $\mathscr{V}$ je vektorov\'y prostor v\v sech omezených a spojitých
funkc\'i na intervalu $\la a,b
\ra.$ Nech?$w(x)$ je kladná funkce na $(a,b)$, pro kterou platí $w(x)\in\LLL(\langle a,b \rangle)$. Pak (\ref{skalarni_soucin})
%\BE \bla f(x)|g(x)\bra_{w}:=\int_a^b f(x)g^\star(x)w(x)~\dx 
%\EE
spl\v nuje axiomy skal\'arn\'iho sou\v cinu na $\mathscr{V}.$


\subsection{Definice}

Spojitou a kladnou funkci $w(x)$ z p\v rede\v slých lemmat naz\'yv\'ame \emph{vahou skal\'arn\'iho sou\v cinu}\index{v\'aha skal\'arn\'iho sou\v cinu} a vybran\'e reprezentanty naz\'yv\'ame n\'asledovn\v e:
%
\begin{itemize}
    \item \emph{standardn\'i (Legendreova) v\'aha:}\index{standardn\'i (Legendreova) v\'aha} pro libovolnou volbu $a,b\in\R$ a $w(x)=\Theta(a)\Theta(b-x),$
    \item \emph{Laguerreova v\'aha:}\index{Laguerreova v\'aha} pro volbu $a=0,$ $b=\infty$ a $w(x)=\Theta(x)\e^{-x},$
    \item \emph{Hermiteova v\'aha:}\index{Hermiteova v\'aha} pro volbu $a=-\infty,$ $b=\infty$ a $w(x)=\e^{-x^2},$
    \item \emph{\v Ceby\v sevova v\'aha:}\index{\v Ceby\v sevova v\'aha} pro volbu $a=-1,$ $b=1$ a $w(x)=\frac{\Theta(1-|x|)}{\sqrt{1-x^2}}.$
\end{itemize}


\subsection{Definice}

Nech\v t $p\geq 1$ je pevn\v e zvolen\'y parametr. Pak t\v r\'idu v\v sech měřitelných funkc\'i $f(\x):G\mapsto\C,$ pro n\v e\v z
%
$$\int_G \bigl|f(\x)\bigr|^p~\dlambda(\x)\in\R,$$
%
ozna\v cujeme symbolem $\LLL_p\bigl(G).$ Neboli

$$\LLL_p\bigl(G)=\left\lbrace f(\x):\Er\mapsto\C:\int_G \bigl|f(\x)\bigr|^p~\dmu(\x)\in\R\right\rbrace$$


\subsection{Věta}

Nechť $f(\x), g(\x)\in \LLL_2(G).$ Potom $f(\x)g^\star(\x)\in\LLL_1(G).$\\

\Proof

\begin{itemize}
\item stačí si uvědomit, že $|f(\x)g^\star(\x)| \leq \frac{1}{2}|f(\x)|^2 + \frac{1}{2}|g(\x)|^2$
\item jelikož oba členy součtu patří do $\LLL(G)$, tak ze srovnávacího kritéria plyne, že také $|f(\x)g^\star(\x)|\in\LLL(G)$
\item je vhodné si zopakovat poznámku 2.1.5 a uvědomit si, že pro měřitelné funkce platí $f(\x)\in\LLL(G)\Leftrightarrow|f(\x)|\in\LLL(G)$
\end{itemize}


\subsection{Pozn\'amka}

Vztahy $\int_G f(x)g^\star(x)w(x)~\dx,$ resp. $\int_G f(\x)g^\star(\x)w(\x)~\dxx$ v\v sak na n\v ekter\'ych vektorov\'ych prostorech skal\'arn\'i sou\v cin nedefinuj\'i. Jedn\'im z takov\'ych prostor\r u je nap\v r. prostor $\LLL_1(0,1).$ Funkce $f(x)=\frac{1}{\sqrt{x}}$ do prostoru $\LLL_1(0,1)$ pat\v r\'i, nebo?
$$\int_0^1\frac{1}{\sqrt{x}}~\dx=2,$$
ale integr\'al
%
$$\int_0^1 \frac{1}{\sqrt{x}}\frac{1}{\sqrt{x}}~\dx=\int_0^1 \frac{1}{x}~\dx$$
%
nekonverguje. Podobn\v e tak\'e prostory $\LLL(G)$ nebo $\LLL_1(G)$ pro $G=(0,\infty)$ negeneruj\'i spolu s operac\'i  $\int_0^\infty f(x)g^\star(x)~\dx$ prehilbertovsk\'y prostor.

$\LLL_2(G)$ také není prehilbertovský, protože není splněn axiom pozitivní definitnosti skalárního součinu, tedy neplatí, že 

$$ \bla f(x)|f(x)\bra=0 \quad \Leftrightarrow \quad f(x)=0$$

Může totiž existovat $f(x)\not=0$ taková, že bude $\int_a^b f(x)f^\star(x)\dx=0$. Například tak, že má nenulovou hodnotu na množině míry nula.


\subsection{Definice}

\emph{Dirichletovou funkc\'i} budeme rozum\v et funkci
%
\BE \Dirichlet(\x):=\left\{\begin{array}{ccl} 1 & \dots & \x\in\Q^r \\ 0 & \dots & \x\in \R^r \setminus \Q^r.
\end{array}\right. \EE


\subsection{Pozn\'amka}

Zavedeme-li na prostoru $\LLL_2(G)$ zobrazen\'i $\la f|g \ra: \LLL_2(G) \times \LLL_2(G) \mapsto\C$ p\v redpisem
%
$$\bla f|g \bra=\int_{G} f(\x)~g^\star(\x)~\dmu(\x),$$
%
pak toto zobrazen\'i nen\'i skal\'arn\'im sou\v cinem, nebo\v t nen\'i spln\v en axiom pozitivn\'i definitnosti z definice skal\'arn\'iho sou\v cinu. Rovnost $\la f|f \ra=0$ by podle n\v eho m\v ela b\'yt spln\v ena tehdy a jen tehdy, pokud $f(
\x)=o(\x),$ tedy pokud $f(\x)$ je ryze nulov\'a funkce. Snadno ale nahl\'edneme, \v ze pro Dirichletovu funkci plat\'i rovnost $\Dirichlet^2(\x)=\Dirichlet(\x),$ a tud\'i\v z (podle teorie Lebesgueova integr\'alu)
%
$$\bla \Dirichlet|\Dirichlet \bra=\int_{G} \Dirichlet(\x)~\Dirichlet^\star(\x)~\dmu(\x)=\int_{G} \Dirichlet(\x)~\dmu(\x)=0.$$
%
Abychom se tedy konečně dostali k nějakému prehilbertovu, a následně Hilbertovu, prostoru budeme potřebovat zobecnění a úvahy, které probereme v následující sekci.

\section{Faktorové prostory funkc\'i, Hilbertovy prostory}

Od termínu funkce nyní přejděme k faktorové funkci, resp. faktorovému prostoru funkcí. Třídu všech funkcí, jež jsou měřitelné a zároveň jsou mezi sebou vzájemně $\mu$-ekvivalentní, tj. liší se pouze na množině míry nula, nazveme \emph{faktorová skupina funkcí}. Třídu všech funkcí, které jsou měřitelné a zároveň ekvivalentní s nulovou funkcí ($f(\x)=0(\x)$) označíme symbolem $\digamma_0$. Do třídy $\digamma_0$ tedy patří i Dirichletova funkce $\Dirichlet(\x).$ Libovolného zástupce z vybrané faktorové skupiny funkci nazveme \emph{faktorovou funkcí}. Pro jednoduchost budeme nadále používat termín funkce, ale mějme pořád na paměti, že jde jen o jednoho vybraného zástupce celé skupiny funkcí.

\subsection{Definice}
\emph{Faktorovou funkc\'i} $\hat{f}(\x)$ nazveme mno\v zinu v\v sech
funkc\'i, je\v z jsou vz\'ajemn\v e $\mu-$ekvivalentn\'i s vybranou m\v e\v
ritelnou funkc\'i $f(\x)\in \Lambda(G),$ tj.
%
$$\hat{f}(\x):=\bigl\{g(\x)\in\Lambda(G):~ g \sim f \bigr\}.$$
%
Mno\v zinu v\v sech faktorov\'ych funkc\'i nazveme
\emph{faktorov\'ym prostorem} nad $G$ a ozna\v c\'ime $\digamma(G).$


\subsection{Pozn\'amka}

Tedy funkce $f(\x)$ a $g(\x)$ z p\v rede\v sl\'e definice se li\v
s\'i pouze na mno\v zin\v e nulov\'e m\'iry. D\'ale si uv\v edomme,
\v ze integr\'al v\v sech prvk\r u faktorov\'e funkce na dan\'e
oblasti $G$ m\'a stejnou hodnotu. M\'a tedy smysl definovat
%
$$\int_G \hat{f}(\x)~\dmu(\x):=\int_G f(\x)~\dmu(\x),$$
%
kde $f(\x)$ je libovoln\'y z\'astupce faktorov\'e funkce
$\hat{f}(\x).$


\subsection{Definice}

Nech\v t $p\geq 1.$ Symbolem $\LL_p(G)$ ozna\v c\'ime mno\v zinu v\v
sech (faktorov\'ych) funkc\'i $f(\x):~G\mapsto \C,$ pro n\v e\v z
$|f(\x)|^p\in\LLL(G),$ tedy
%
$$\int_G |f(\x)|^p~\dmu(\x)<+\infty.$$


\subsection{Věta}

Zobrazen\'i $\la f|g \ra: \LL_2(G) \times \LL_2(G) \mapsto\C$ zaveden\'e na $\LL_2(G)$ p\v redpisem
%
\BE \bla f|g \bra=\int_{G} f(\x)~g^\star(\x)~\dmu(\x) \label{skalarnisoucin} \EE
%
reprezentuje skal\'arn\'i sou\v cin. Prostor $\LL_2(G)$ je tud\'i\v z prehilbertovsk\'ym prostorem.\\

\Proof

\begin{itemize}
\item axiom lev\'e linearity je spln\v en trivi\'aln\v e, podobn\v e jako hermiticita
\item pro libovolnou funkci  $f(\x)\in \LL_2(G)$ pak plat\'i, \v ze
%
$$\bla f|f \bra=\int_{G} f(\x)~f^\star(\x)~\dmu(\x)=\int_{G} |f(\x)|^2~\dmu(\x) \geq 0$$
%
a nav\'ic rovnost
%
$$\bla f|f \bra=\int_{G} f(\x)~f^\star(\x)~\dmu(\x)=\int_{G} |f(\x)|^2~\dmu(\x) = 0$$
%
nast\'av\'a pouze pro nulovou faktorou funkci
\item t\'im je napln\v en axiom pozitivn\'i definitnosti
\item zb\'yv\'a dok\'azat, \v ze pro libovoln\'e dv\v e funkce  $f(\x),g(\x)\in \LL_2(G)$ je v\'yraz $\bla f|g \bra=\int_{G} f(\x)~g^\star(\x)~\dmu(\x)$ dob\v re definov\'an
\item jeliko\v z je na $G$ spln\v ena nerovnost $$2\bigl|f(\x)g^\star(\x)\bigr|\leq \bigl|f(\x)\bigr|^2+\bigl|g^\star(\x)\bigr|^2=\bigl|f(\x)\bigr|^2+\bigl|g(\x)\bigr|^2$$ a oba integr\'aly $\int_{G} \bigl|f(\x)\bigr|^2~\dlambda(\x)$ a $\int_{G} \bigl|g(\x)\bigr|^2~\dlambda(\x)$ existuj\'i z definice prostoru $\LL_2(G)$ a z v\v ety o absolutn\'i hodnot\v e Lebesgueova integr\'alu, existuje podle srovn\'avac\'iho krit\'eria tak\'e integr\'al $\int_{G} f(\x)g^\star(\x)~\dmu(\x)$

\end{itemize}


\subsection{Pozn\'amka}

Je-li vztah (\ref{skalarnisoucin}) skal\'arn\'im sou\v cinem na $\LL_2(G),$ pak je zobrazen\'i
%
$$\bigl\|f(\x)\bigr\|=\sqrt{\int_G \bigl|f(\x)\bigr|^2~\dlambda(\x)}$$
%
normou na $\LL_2(G).$  Zobrazen\'i 
%
$$\rho(f,g):=\sqrt{\int_G \bigl|f(\x)-g(\x)\bigr|^2 ~\dlambda(\x)}$$
%
 je metrikou na $\LL_2(G).$

\subsection{Definice}

\v Rekneme, \v ze posloupnost funkc\'i $\bigl(f_n(\x)\bigr)_{n=1}^\infty$ z prostoru $\LL_2(G)$ \emph{konverguje podle normy}\index{konvergence podle normy} k funkci $f(\x)\in\LL_2(G),$ pokud pro ka\v zd\'e $\ep>0$ existuje $n_0\in\N$ tak, \v ze pro v\v sechna $n\geq n_0$ plat\'i
%
$$\bigl\|f_n(\x)-f(\x)\bigr\|<\ep,$$
%
to jest
%
$$\sqrt{\int_G \bigl|f_n(\x)-f(\x)\bigr|^2 ~\dmu(\x)}<\ep.$$
%
Konvergenci podle normy zapisujeme symbolem $f_n(\x) \nkonv f(\x).$


\subsection{P\v r\'iklad}

Rozhodn\v eme podle definice, zda posloupnost funkc\'i $\bigl(\e^{-nx^2}\bigr)_{n=1}^\infty$ z prostoru $\LL_2(\R)$ konverguje podle normy k nulov\'e funkci. Nech\v t $\ep>0$ je zvoleno libovoln\v e. Limitn\'i faktorovou funkc\'i pro zkoumanou posloupnost je nulov\'a funkce. Zkoumejme tedy nerovnost $$\bigl\|\e^{-nx^2}\bigr\|=\sqrt{\int_G \e^{-2nx^2} ~\dmu(\x)}=\left(\frac{\pi}{2n}\right)^{1/4}<\ep.$$ Za hledan\'e $n_0\in\N$ z definice konvergence podle normy tedy sta\v c\'i volit $$n_0:=\left\lfloor\frac{\pi}{2\ep^4}\right\rfloor+1.$$ Pov\v simn\v eme si ale paradoxu, \v ze posloupnost $\bigl(\e^{-nx^2}\bigr)_{n=1}^\infty$ nekonverguje (uva\v zujeme-li konvergenci klasickou) k nulov\'e funkci ani stejnom\v ern\v e ani bodov\v e. Vztah mezi klasickou konvergenc\'i a konvergenc\'i podle normy lze shrnout v n\'asleduj\'ic\'i v\v et\v e.

\subsection{V\v eta}

Nech\v t je d\'ana posloupnost funkc\'i $\bigl(f_n(\x)\bigr)_{n=1}^\infty$ z prostoru $\LL_2(G)$ takov\'a, \v ze $f_n(\x) \stackrel{G}{\rightrightarrows} f(\x) \in \LL_2(G).$ Nech\v t d\'ale $0<\mu(G)<\infty.$ Pak $f_n(\x) \nkonv f(\x).$\\

\Proof

\begin{itemize}
\item z p\v redpoklad\r u plyne, \v ze  pro v\v sechna $\widetilde{\ep} >0$ existuje $n_0$ tak, \v ze pro
v\v sechna $n \geq n_0$ a pro v\v sechna $\x\in G$ plat\'i nerovnost $$\bigl| f_n(\x)-f(\x) \bigr| < \widetilde{\ep} = \frac{\ep}{\sqrt{4\mu(G)}}$$
\item jeliko\v z zjevn\v e $$\bigl\|f_n(\x)-f(\x)\bigr\|^2 = \bla f_n-f|f_n-f \bra = \int_G \bigl| f_n(\x)-f(\x) \bigr|^2~\dmu(\x) \leq \frac{\ep^2}{4\mu(G)} \mu(G) = \frac{\ep^2}{4},$$ zji\v s\v tujeme, \v ze pro indexy $n\geq n_0$ plat\'i nerovnost $\bigl\|f_n(\x)-f(\x)\bigr\| \leq \frac{\ep}{2} < \ep$
\item to dokazuje skute\v cnost, \v ze posloupnost funkc\'i $\bigl(f_n(\x)\bigr)_{n=1}^\infty$  konverguje podle normy k funkci $f(\x)$
\end{itemize}

\subsection{V\v eta}

Nech\v t $f_n(\x) \nkonv f(\x).$ Pak existuje podposloupnost $\bigl(f_{k_n}(\x)\bigr)_{n=1}^\infty$ vybran\'a z posloupnosti $\bigl(f_n(\x)\bigr)_{n=1}^\infty$ takov\'a, \v ze plat\'i $f_{k_n}(\x) \rightarrow f(\x)$ skoro v\v sude v $M.$\\

\Proof

\begin{itemize}
\item viz \textcolor{red}{odkázat se na zdroj}, str. 42, p\v r\'iklad 2.2.2
\end{itemize}


\subsection{Definice}

Nech\v t je d\'an vektorov\'y prostor $\V$ se skal\'arn\'im sou\v
cinem $\la.|.\ra.$ Nech\v t $\|.\|$ je norma generovan\'a
zadan\'ym skal\'arn\'im sou\v cinem a $\rho(x,y)$ metrika
generovan\'a v\'y\v se uvedenou normou. Nech\v t nav\'ic
$\{\V,\rho\}$ je \'upln\'ym metrick\'ym prostorem. Pak takov\'y
prostor $\H:=\bigl\{\V,\la.|.\ra,\|.\|,\rho\bigr\}$ naz\'yv\'ame
\emph{Hilbertov\'ym}\index{Hilbert\r uv prostor} prostorem.

\subsection{Pozn\'amka}

Metrick\'y prostor $\{M,\rho\}$ s libovolnou metrikou $\rho(f,g)$
nazveme \emph{\'upln\'ym,}\index{\'upln\'y metrick\'y prostor}
jestli\v ze ka\v zd\'a cauchyovsk\'a posloupnost je v~~n\v em
konvergentn\'i.


\subsection{V\v eta -- \emph{o spojitosti skal\'arn\'iho sou\v cinu}}\index{v\v eta o spojitosti skal\'arn\'iho sou\v cinu}\label{Hilbert_so_beautiful}

Nech\v t je d\'an Hilbert\r uv prostor $\H$ nad t\v elesem $\C.$  Nech\v t je d\'ana posloupnost funkc\'i $(f_n(\x))_{n=1}^\infty$ z prostoru $\H,$ kter\'a konverguje podle normy k funkci $f(\x)\in\H,$ a funkce $g(\x)\in\H.$ Pak plat\'i
%
$$\limn \la f_n|g \ra = \la f|g \ra, \quad \limn \la g|f_n \ra = \la g|f \ra.$$

\Proof

\begin{itemize}
\item jedn\'a se o bezprost\v redn\'i d\r usledek v\v ety \ref{love_so_beautiful}
\end{itemize}

\subsection{Definice}\label{kosta}

\v Rekneme, \v rada funkc\'i $\sum_{n=1}^\infty f_n(\x)$ z Hilbertova prostoru $\H$ \emph{konverguje podle normy}\index{konvergence \v rady podle normy} ke sv\'emu sou\v ctu $s(\x)\in\H,$ pokud posloupnost  $\bigl(s_n(\x)\bigr)_{n=1}^\infty$ jej\'ich \v c\'aste\v cn\'ych sou\v ct\r u
%
$$s_n(\x):=\sum_{k=1}^n f_k(\x)$$
%
konverguje podle normy k funkci $s(\x),$ tj. $\limnorm_{n\rightarrow \infty} s_n(\x)=s(\x).$ Konvergenci podle normy zapisujeme symbolem $\sumline_{n=1}^\infty f_n(\x) = s(\x).$ 



\subsection{V\v eta}

Faktorov\'y prostor $\LL_2(G)$ spole\v cn\v e se skal\'arn\'im
sou\v cinem zaveden\'ym vztahem (\ref{skalarnisoucin}) je \'upln\'y, tj. jedn\'a se o Hilbert\r uv prostor.\\

\Proof\\

Jeliko\v z ji\v z bylo prok\'az\'ano, \v ze $\LL_2(G)$ je
vektorov\'y prostor nad $\C,$ zb\'yv\'a dok\'azat \'uplnost.
Vyberme tedy z~~libovoln\'e cauchyovsk\'e posloupnosti
$\bigl(f_k(\x)\bigr)_{k=1}^\infty$ podposloupnost
$\bigl(f_{k_\ell}(\x)\bigr)_{\ell=1}^\infty,$ je\v z konverguje
skoro v\v sude na $G.$ To je d\'iky cauchyovskosti mo\v zn\'e.
C\'ilem d\r ukazu je de facto prok\'azat, \v ze
$\bigl(f_k(\x)\bigr)_{k=1}^\infty$ je konvergentn\'i v
$\LL_2(G).$ Prvn\'i \v clen podposloupnosti
$\bigl(f_{k_\ell}(\x)\bigr)_{\ell=1}^\infty$ vyberme tak, aby pro
v\v sechna $m>k_1$ platilo
%
$$\|f_{k_1}(\x)-f_m(\x)\| < \frac{1}{2}.$$
%
To je op\v et d\'iky cauchyovskosti mo\v zn\'e. Druh\'y \v clen podposloupnosti vyberme tak,
aby pro v\v sechna $m>k_2$ platilo
%
$$\|f_{k_2}(\x)-f_m(\x)\| < \frac{1}{2^2}.$$
%
Analogicky vyberme $\ell-$t\'y \v clen podposloupnosti tak, aby
pro v\v sechna $m>k_\ell$ platilo
%
$$\|f_{k_\ell}(\x)-f_m(\x)\| < \frac{1}{2^\ell}.$$
%
Ozna\v c\'ime-li nyn\'i
%
$$g_k(\x)=\sum_{s=1}^k \bigl|f_{k_s+1}(\x)-f_{k_s}(\x)\bigr|,$$

$$g(\x)=\sum_{s=1}^\infty \bigl|f_{k_s+1}(\x)-f_{k_s}(\x)\bigr|,$$
%
bude
%
$$\|g_k(\x)\|\leq \sum_{s=1}^k
\bigl\|f_{k_s+1}(\x)-f_{k_s}(\x)\bigr\| < \sum_{s=1}^k
\frac{1}{2^s} <1.$$
%
Je proto $\int_M |g_n(\x)|^2~\dmu(\x)<1$ a podle Leviho v\v ety
tak\'e
%
$$\int_M |g(\x)|^2~\dmu(\x)=\limk \int_M |g_k(\x)|^2~\dmu(\x) \leq
1$$
%
a $g(\x)$ je kone\v cn\'a skoro v\v sude na $M.$ Nav\'ic \v rada
$\sum_{s=1}^k \bigl|f_{k_s+1}(\x)-f_{k_s}(\x)\bigr|$ m\'a pro
skoro v\v sechna $\x\in M$ kone\v cn\'y sou\v cet  a tud\'i\v z i
\v rada $\sum_{s=1}^k \bigl(f_{k_s+1}(\x)-f_{k_s}(\x)\bigr)$ je
konvergentn\'i, a tedy tak\'e posloupnost
%
$$f_{k}(\x)=\sum_{s=1}^{k-1}
\bigl(f_{k_s+1}(\x)-f_{k_s}(\x)\bigr)+f_{k_1}(\x).$$
%
Ozna\v cme $f(\x)$ jej\'i limitu. Ta je samoz\v rejm\v e m\v e\v
riteln\'a jako limita posloupnosti m\v e\v riteln\'ych funkc\'i.\\

Ve druh\'e \v c\'asti d\r ukazu uk\'a\v zeme, \v ze posloupnost
$\bigl(f_{k_\ell}(\x)\bigr)_{k=1}^\infty$  konverguje pr\'av\v e k
t\'eto funkci $f(\x)$ v $\LL_2(G).$ P\v redn\v e z cauchyovskosti
posloupnosti $\bigl(f_{k_\ell}(\x)\bigr)_{\ell=1}^\infty$  plyne
cauchyovskost podposloupnosti
$\bigl(f_{k_\ell}(\x)\bigr)_{k=1}^\infty,$ a tedy pro $\epsilon=1$
existuje $k_0\in\N$ takov\'e, \v ze pro $\ell>k_0$ a $m>k_0$ je
%
$$\int_M \bigl|f_{k_\ell}(\x)-f_{k_m}(\x)\bigr|^2~\dmu(\x)<1.$$
%
Podle Fatouovy v\v ety (viz v\v eta 2.1.7, str. 26 v (někde - \textcolor{red}{DOPLNIT}) je
%
$$\int_M \bigl|f_{k_\ell}(\x)-f(\x)\bigr|^2~\dmu(\x)<1,$$
%
odkud plyne, \v ze funkce $f(\x)$ rozepsan\'a jako
$\bigl(f(\x)-f_{k_\ell}(\x)\bigr)+f_{k_\ell}(\x)$ pat\v r\'i do
$\LL_2(G).$ Provedeme-li stejnou \'uvahu s libovoln\v e mal\'ym
$\epsilon,$ z\'isk\'ame
%
$$\int_M \bigl|f_{k_\ell}(\x)-f(\x)\bigr|^2~\dmu(\x)<\epsilon^2,$$
%
co\v z neznamen\'a nic jin\'eho, ne\v z \v ze
%
$$\lim_{\ell \rightarrow \infty} f_{k_\ell}(\x)=f(\x).$$

V posledn\'i \v c\'asti d\r ukazu uk\'a\v zeme, \v ze k funkci
$f(\x)$ konverguje cel\'a posloupnost
$\bigl(f_k(\x)\bigr)_{k=1}^\infty.$ To ov\v sem plyne ihned z
nerovnost\'i
%
$$\bigl\|f(\x)-f_k(\x)\bigr\| \leq
\bigl\|f(\x)-f_{k_\ell}(\x)\bigr\| +
\bigl\|f_{k_\ell}(\x)-f_k(\x)\bigr\|,$$
%
nebo\v t prvn\'i \v clen napravo m\r u\v zeme ud\v elat libovoln\v
e mal\'ym (pro velk\'a $k_\ell$) d\'iky dok\'azan\'e konvergenci
zmi\v novan\'e podposloupnosti a druh\'y d\'iky cauchyovskosti
posloupnosti $\bigl(f_k(x)\bigr)_{k=1}^\infty.$

\subsection{D\r usledek}

Nech\v t $w(\x)\in\CC(G)$ je kladná funkce. Faktorov\'y prostor
%
$$\LL_2^{(w)}(G)=\bigl\{f(\x)\in\digamma(G):~\int_G |f(\x)|^2 w(\x)~\dmu(\x)<+\infty\bigr\},$$
%
spole\v cn\v e se skal\'arn\'im sou\v cinem zaveden\'ym vztahem  $\int_G f(\x)g^\star(\x)w(\x)~\dxx$ je Hilbertov\'ym prostorem.\\

%%%%%%%%%%%%%%%%%%%%%%%%%%%%%%%%%%%%%%%%%%%%%%%%%%%%%%%%%%%%%%%%%%%%%%%%%%%%%%%%%%%%%%%%%
%%%%%%%%%%%%%%%%%%%%%%%%%%%%%%%%%%%%%%%%%%%%%%%%%%%%%%%%%%%%%%%%%%%%%%%%%%%%%%%%%%%%%%%%%
%%%%%%%%%%%%%%%%%%%%%%%%%%%%%%%%%%%%%%%%%%%%%%%%%%%%%%%%%%%%%%%%%%%%%%%%%%%%%%%%%%%%%%%%%
%Zde jsem zacal ja - Petr
\subsection{Věta}

$f(\x) \in \LLL_2(G) \wedge H \subset G \wedge \mu(H)<+\infty \quad \Rightarrow \quad f(\x)\in \LLL_1(H)$\\

\Proof

\begin{itemize}
\item chceme $\int_H |f(\x)|~\dxx \in \R$
\item $\int_H |f(\x)|~\dxx = \int_G |f(\x)|\chi_H(\x)~\dxx \leq \overbrace{\dfrac{1}{2}\int_G |f(\x)|^2~\dxx}^{\in \R} + \overbrace{\dfrac{1}{2}\int_G \chi_H(\x)~\dxx}^{\dfrac{1}{2}\lambda(H)}$
\end{itemize}

\subsection{Důsledek}

Pro $H\in\Mla$, pro kter\'e $\mu(H) < \infty$, plat\'i $\LLL_2(H)\overset{\neq}{\subset}\LLL_1(H)$
\textcolor{red}{Znak pro nerovnou inkluzi}

\subsection{Pozn\'amka}
V\'ime, \v ze $\CC(\la a,b\ra)$ je prehilbertovsk\'ym prostorem a \v ze skal\'arn\'i sou\v cin je definov\'an integr\'alem $\int_a^b f(x)g^\star(x)w(x)\dx = \la f|g\ra_w$. Zkoumejme, je-li tak\'e prostorem Hilbertovsk\'ym.

\textcolor{red}{Doplnit obrazek}

Tato posloupnost, a\v ckoli je Cauchyovsk\'a, (viz obr\'azek) nem\'a limitu v $\CC(\la a,b\ra)$, co\v z je spor s definic\'i limity. T\'im p\'adem je $\CC(\la a,b\ra)$ ne\'upln\'ym, tedy nehilbertovsk\'ym prostorem.

\subsection{P\v r\'iklad}
Posloupnost $a_n=\Bigl(1+\dfrac{1}{n}\Bigr)^n \in \Q$. Posloupnost $\(a_n)_{n=1}^\infty$ je Cauchyovsk\'a, ale $\limn a_n \overset{\Q}{=}$ neexistuje.

\subsection{V\v eta}

Faktorov\'y prostor $\LL_2(G)$ spole\v cn\v e se skal\'arn\'im
sou\v cinem zaveden\'ym vztahem (\ref{skalarnisoucin}) je \'upln\'y, tj. jedn\'a se o Hilbert\r uv prostor.\\

\Proof\\

Jeliko\v z ji\v z bylo prok\'az\'ano, \v ze $\LL_2(G)$ je
vektorov\'y prostor nad $\C,$ zb\'yv\'a dok\'azat \'uplnost.
Vyberme tedy z~~libovoln\'e cauchyovsk\'e posloupnosti
$\bigl(f_k(\x)\bigr)_{k=1}^\infty$ podposloupnost
$\bigl(f_{k_\ell}(\x)\bigr)_{\ell=1}^\infty,$ je\v z konverguje
skoro v\v sude na $G.$ To je d\'iky cauchyovskosti mo\v zn\'e.
C\'ilem d\r ukazu je de facto prok\'azat, \v ze
$\bigl(f_k(\x)\bigr)_{k=1}^\infty$ je konvergentn\'i v
$\LL_2(G).$ Prvn\'i \v clen podposloupnosti
$\bigl(f_{k_\ell}(\x)\bigr)_{\ell=1}^\infty$ vyberme tak, aby pro
v\v sechna $m>k_1$ platilo
%
$$\|f_{k_1}(\x)-f_m(\x)\| < \frac{1}{2}.$$
%
To je op\v et d\'iky cauchyovskosti mo\v zn\'e. Druh\'y \v clen podposloupnosti vyberme tak,
aby pro v\v sechna $m>k_2$ platilo
%
$$\|f_{k_2}(\x)-f_m(\x)\| < \frac{1}{2^2}.$$
%
Analogicky vyberme $\ell-$t\'y \v clen podposloupnosti tak, aby
pro v\v sechna $m>k_\ell$ platilo
%
$$\|f_{k_\ell}(\x)-f_m(\x)\| < \frac{1}{2^\ell}.$$
%
Ozna\v c\'ime-li nyn\'i
%
$$g_k(\x)=\sum_{s=1}^k \bigl|f_{k_s+1}(\x)-f_{k_s}(\x)\bigr|,$$

$$g(\x)=\sum_{s=1}^\infty \bigl|f_{k_s+1}(\x)-f_{k_s}(\x)\bigr|,$$
%
bude
%
$$\|g_k(\x)\|\leq \sum_{s=1}^k
\bigl\|f_{k_s+1}(\x)-f_{k_s}(\x)\bigr\| < \sum_{s=1}^k
\frac{1}{2^s} <1.$$
%
Je proto $\int_M |g_n(\x)|^2~\dmu(\x)<1$ a podle Leviho v\v ety
tak\'e
%
$$\int_M |g(\x)|^2~\dmu(\x)=\limk \int_M |g_k(\x)|^2~\dmu(\x) \leq
1$$
%
a $g(\x)$ je kone\v cn\'a skoro v\v sude na $M.$ Nav\'ic \v rada
$\sum_{s=1}^k \bigl|f_{k_s+1}(\x)-f_{k_s}(\x)\bigr|$ m\'a pro
skoro v\v sechna $\x\in M$ kone\v cn\'y sou\v cet  a tud\'i\v z i
\v rada $\sum_{s=1}^k \bigl(f_{k_s+1}(\x)-f_{k_s}(\x)\bigr)$ je
konvergentn\'i, a tedy tak\'e posloupnost
%
$$f_{k}(\x)=\sum_{s=1}^{k-1}
\bigl(f_{k_s+1}(\x)-f_{k_s}(\x)\bigr)+f_{k_1}(\x).$$
%
Ozna\v cme $f(\x)$ jej\'i limitu. Ta je samoz\v rejm\v e m\v e\v
riteln\'a jako limita posloupnosti m\v e\v riteln\'ych funkc\'i.\\

Ve druh\'e \v c\'asti d\r ukazu uk\'a\v zeme, \v ze posloupnost
$\bigl(f_{k_\ell}(\x)\bigr)_{k=1}^\infty$  konverguje pr\'av\v e k
t\'eto funkci $f(\x)$ v $\LL_2(G).$ P\v redn\v e z cauchyovskosti
posloupnosti $\bigl(f_{k_\ell}(\x)\bigr)_{\ell=1}^\infty$  plyne
cauchyovskost podposloupnosti
$\bigl(f_{k_\ell}(\x)\bigr)_{k=1}^\infty,$ a tedy pro $\epsilon=1$
existuje $k_0\in\N$ takov\'e, \v ze pro $\ell>k_0$ a $m>k_0$ je
%
$$\int_M \bigl|f_{k_\ell}(\x)-f_{k_m}(\x)\bigr|^2~\dmu(\x)<1.$$
%
Podle Fatouovy v\v ety (viz v\v eta 2.1.7, str. 26 v \cite{Krbalek_RMF}) je
%
$$\int_M \bigl|f_{k_\ell}(\x)-f(\x)\bigr|^2~\dmu(\x)<1,$$
%
odkud plyne, \v ze funkce $f(\x)$ rozepsan\'a jako
$\bigl(f(\x)-f_{k_\ell}(\x)\bigr)+f_{k_\ell}(\x)$ pat\v r\'i do
$\LL_2(G).$ Provedeme-li stejnou \'uvahu s libovoln\v e mal\'ym
$\epsilon,$ z\'isk\'ame
%
$$\int_M \bigl|f_{k_\ell}(\x)-f(\x)\bigr|^2~\dmu(\x)<\epsilon^2,$$
%
co\v z neznamen\'a nic jin\'eho, ne\v z \v ze
%
$$\lim_{\ell \rightarrow \infty} f_{k_\ell}(\x)=f(\x).$$

V posledn\'i \v c\'asti d\r ukazu uk\'a\v zeme, \v ze k funkci
$f(\x)$ konverguje cel\'a posloupnost
$\bigl(f_k(\x)\bigr)_{k=1}^\infty.$ To ov\v sem plyne ihned z
nerovnost\'i
%
$$\bigl\|f(\x)-f_k(\x)\bigr\| \leq
\bigl\|f(\x)-f_{k_\ell}(\x)\bigr\| +
\bigl\|f_{k_\ell}(\x)-f_k(\x)\bigr\|,$$
%
nebo\v t prvn\'i \v clen napravo m\r u\v zeme ud\v elat libovoln\v
e mal\'ym (pro velk\'a $k_\ell$) d\'iky dok\'azan\'e konvergenci
zmi\v novan\'e podposloupnosti a druh\'y d\'iky cauchyovskosti
posloupnosti $\bigl(f_k(x)\bigr)_{k=1}^\infty.$

\subsection{D\r usledek}

Nech\v t $w(\x)\in\CC(G)$ je nenulov\'a a nez\'aporn\'a funkce. Faktorov\'y prostor
%
$$\LL_w(G)=\bigl\{f(\x)\in\digamma(G):~\int_G |f(\x)|^2 w(\x)~\dmu(\x)<+\infty\bigr\},$$
%
kde $0<w(\x)\in\CC(G),$ spole\v cn\v e se skal\'arn\'im sou\v cinem zaveden\'ym vztahem  $\int_G f(\x)g^\star(\x)w(\x)~\dxx$ je Hilbertov\'ym prostorem.\\

\subsection{P\v r\'iklad}
Skal\'arn\'i sou\v ciny na funkcion\'aln\'ich vektorov\'ych prostorech jednorozm\v ern\'ych funkc\'i
\begin{itemize} \item Legendre $\Theta(x-a)\Theta(b-x),$ $G=(a,b)$  \item Laguerre $\Theta(x)\e^{-x},$ $G=(0,+\infty)$ \item Hermite $\e^{-x^2},$ $G=\R$
\end{itemize}
\textcolor{red}{dualita s 2.2.9}

\subsection{Definice}

\v Rekneme, \v ze funkce $f(\x):\Er\mapsto\R$ je analytick\'a na $G,$ jestli\v ze pro ka\v zd\'e $\ccc\in G$ existuje okol\'i $\U_\ep(\ccc)$ tak, \v ze pro v\v sechna $\x\in \U_\ep(\ccc)$ plat\'i rovnost
%
$$f(\x)=\sum_{n=0}^{\infty} \frac{\d^n f_{\ccc} (\x) }{n!},$$
%
kde symbol $\d^n f_{\ccc} (\x)$ p\v redstavuje $n-$t\'y tot\'aln\'i diferenci\'al v bod\v e funkce $f(\x)$ v bod\v e $\ccc.$ T\v r\'idu v\v sech analytick\'ych funkc\'i na oblasti $G$ ozna\v cujeme symbolem $\A_G.$ \textcolor{red}{Tady mi to nesedi s poznamkami, overit.}

\subsection{Zna\v cen\'i}
Nad\'ale budeme zna\v cit funkcion\'aln\'i Hilbert\r uv prostor symbolem $\H$, p\v ri\v cem\v z p\v redpokl\'ad\'ame prostor faktorov\'ych funkc\'i $\LL_2$ nebo $\LL_2^{(w)}$.

\subsection{Pozn\'amka}
Rozli\v sujeme 3 typy konvergence:
\begin{itemize}
\item bodov\'a: $f_n(\x)\overset{G}{\bkonv}f(\x)$
\item stejnom\v ern\'a: $f_n(\x)\overset{G}{\skonv}f(\x)\Leftrightarrow (\forall \ep >0)(\exists n_0\in\N)(n>n_0 \wedge\x\in G\Rightarrow |f_n(\x)-f(\x)|<\ep)$
\item podle normy: $f_n(\x)\nkonv f(\x) \Leftrightarrow (\forall \ep >0)(\exists n_0\in\N)(n>n_0 \wedge \|f_n(\x)-f(\x)\|<\ep)$  
\end{itemize}
Poznamenejme, \v ze ve v\'yrazu $\|f_n(\x)-f(\x)\|$ je skryt Lebesgue\r uv integr\'al, kter\'y konverguje i tam, kde na mno\v zin\v e nulov\'e m\'iry nekonverguje.

\subsection{V\v eta}
Nech\v t $f_n(\x)\overset{\langle a,b\rangle}{\skonv}f(\x)$ a $\langle f|g\rangle_w$ je skal\'arn\'i sou\v cin dle definice \ref{skalarnisoucin} pro $\CC(\langle a,b\rangle)$. Pak $f_n(\x)\nkonv(\x)$.\\

\NoProof

\subsection{V\v eta}
$f_n(\x)\overset{G}{\skonv}f(\x) \wedge \mu(G)<+\infty \Rightarrow f_n(\x)\overset{\LLL_2^{(w)}(G)}{\nkonv}f(\x).$ Je-li v\'aha omezen\'a na $G.$\\

\Proof
\begin{itemize}
\item $\|f_n(\x)-f(\x)\|^2=\la f_n(\x)-f(\x)|f_n(\x)-f(\x)\ra=\int_G |f_n(\x)-f(\x)|^2~\dxx\leq\mu(G)\sup |f_n(\x)-f(\x)|^2\leq \dfrac{1}{2}\ep<\ep$
\item $(\forall\ep>0)(\exists n_0\in\N)\Bigg((n\geq n_0 \wedge \x\in G)\Rightarrow |f_n(\x)-f(\x)|<\sqrt{\dfrac{\ep}{2\mu(G)}}\Bigg)$
\end{itemize}

\subsection{V\v eta -- \emph{o spojitosti skal\'arn\'iho sou\v cinu}}\index{v\v eta o spojitosti skal\'arn\'iho sou\v cinu}\label{Hilbert_so_beautiful}

Nech\v t je d\'an Hilbert\r uv prostor $\H$ nad t\v elesem $\C.$  Nech\v t je d\'ana posloupnost funkc\'i $(f_n(\x))_{n=1}^\infty$ z prostoru $\H,$ kter\'a konverguje podle normy k funkci $f(\x)\in\H,$ a funkce $g(\x)\in\H.$ Pak plat\'i
%
$$\limn \la f_n|g \ra = \la f|g \ra, \quad \limn \la g|f_n \ra = \la g|f \ra.$$

\Proof
\begin{itemize}
\item $|\la f_n(\x)|g(\x) \ra - \la f(\x)|g(\x) \ra| = |\la f_n(\x)-f(\x)|g(\x) \ra| \leq \|f_n(\x)-f(\x)\|\cdot\|g(\x)\|<\ep$
\item $\|f_n(\x)-f(\x)\|<\dfrac{\ep}{\|g(\x)\|}$
\item Toto plat\'i ve v\v sech p\v ripadech krom\v e $\|g(\x)\|=0$, co\v z je v\v sak trivi\'aln\'i
\end{itemize}


\subsection{Definice}
Nech\v t $\sum_{n=1}^{+\infty}{f_n(\x)}$ je funkcion\'aln\'i \v rada, a $s_n(\x)$ jej\'i \v n-t\'y c\'aste\v cn\'y sou\v cet. Pak ozna\v c\'ime:

\begin{itemize}
\item sou\v cet podle normy $\sumline_{n=1}^{+\infty}{f_n(\x)}$
\item limitu podle normy $\limnormn s_n(\x)=s(\x)\in\H$
\item tedy $\sumline_{n=1}^{+\infty}{f_n(\x)}=s(\x)\in\H$
\end{itemize}


\subsection{V\v eta}
$f_n(\x), s(\x), g(\x)\in\H $ a $\sumline_{n=1}^{+\infty}{f_n(\x)}=s(\x)$. Pak plat\'i
\[
\sum_{n=1}^{+\infty}\la f_n(\x)|g(\x) \ra = \la s(\x)|g(\x) \ra \equiv \Bigg\langle \sumline_{n=1}^{+\infty} f_n(\x)|g(\x)\Bigg\rangle.
\] 

\Proof\\
\textcolor{red}{dodelam (prejit na posloupnost castecnych souctu)}

\subsection{P\v r\'iklad}
Domácí úkol: Je $\|f\|_\infty:=\max_{x\in\langle a,b\rangle}|f|$ normou na $\CC(\langle a,b\rangle)?$ A je $\CC(\langle a,b\rangle)$ s touto normou \'upln\'y?


